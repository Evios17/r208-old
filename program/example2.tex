\documentclass{article}
\usepackage{multicol}
\usepackage{array}
\usepackage{makeidx}
\usepackage[skaknew]{chessfss}
\usepackage{texmate}
\usepackage{xskak}
\usepackage[top=1.5cm, bottom=2cm, left=1.5cm, right=1cm,headheight=15pt]{geometry}
\usepackage{adjmulticol}
\usepackage{ragged2e}


\begin{document}

\chessevent{Hastings}
\chessopening{Hastings ENG}

Date : 1895.08.17
EventDate : 1895.08.05
Round : 10
Result : 1-0
\whitename{Wilhelm Steinitz}
\blackname{Curt von Bardeleben}
\makegametitle
\begin{multicols}{2}
\noindent
\newchessgame[id=main]
\xskakset{style=styleC}
\mainline{1. e4 e5 2. Nf3 Nc6 3. Bc4 Bc5 4. c3 Nf6 5. d4 exd4 }
\scalebox{0.90}{\chessboard}
\mainline{6. cxd4 Bb4+ 7. Nc3 d5 8. exd5 Nxd5 9. O-O Be6 10. Bg5 Be7 }
\xskakcomment{\small\texttt\justifying{\textcolor{darkgray}{~ 
[pgndiagram] Sharp gambits such as this one were quite popular in the 19th
century.; Not a critical reply; Now White has the initiative;  }}}

\scalebox{0.90}{\chessboard}
\mainline{11. Bxd5 Bxd5 12. Nxd5 Qxd5 13. Bxe7 Nxe7 14. Re1 f6 15. Qe2 Qd7 }
\xskakcomment{\small\texttt\justifying{\textcolor{darkgray}{~ 
[pgndiagram] Sharp gambits such as this one were quite popular in the 19th
century.; Not a critical reply; Now White has the initiative; [pgndiagram] A very counterintuitive
exchange. You can feel the understanding Steinitz had for the game from moves
like these. His games are full of deep strategic ideas.; The right
recapture; Once again the
right choice; 
[pgndiagram] If only Black could castle, he would have no problems whatsoever.
As it is, White has the initiative;  }}}

\scalebox{0.90}{\chessboard}
\mainline{16. Rac1 c6 17. d5 cxd5 18. Nd4 Kf7 19. Ne6 Rhc8 20. Qg4 g6 }
\xskakcomment{\small\texttt\justifying{\textcolor{darkgray}{~ 
[pgndiagram] Sharp gambits such as this one were quite popular in the 19th
century.; Not a critical reply; Now White has the initiative; [pgndiagram] A very counterintuitive
exchange. You can feel the understanding Steinitz had for the game from moves
like these. His games are full of deep strategic ideas.; The right
recapture; Once again the
right choice; 
[pgndiagram] If only Black could castle, he would have no problems whatsoever.
As it is, White has the initiative; [pgndiagram] So far the game
has been of a rather high quality. This move releases the tension; And immediately Bardeleben returns the favour; [pgndiagram] A thematic clearance move.Now White is winning. The
way Steinitz concludes the game is highly spectacular; The
knight is coming to e6 with great effect; Now the g7 also is weak;  }}}

\scalebox{0.90}{\chessboard}
\mainline{21. Ng5+ Ke8 22. Rxe7+ Kf8 23. Rf7+ Kg8 24. Rg7+ Kh8 25. Rxh7+ Kg8 }
\xskakcomment{\small\texttt\justifying{\textcolor{darkgray}{~ 
[pgndiagram] Sharp gambits such as this one were quite popular in the 19th
century.; Not a critical reply; Now White has the initiative; [pgndiagram] A very counterintuitive
exchange. You can feel the understanding Steinitz had for the game from moves
like these. His games are full of deep strategic ideas.; The right
recapture; Once again the
right choice; 
[pgndiagram] If only Black could castle, he would have no problems whatsoever.
As it is, White has the initiative; [pgndiagram] So far the game
has been of a rather high quality. This move releases the tension; And immediately Bardeleben returns the favour; [pgndiagram] A thematic clearance move.Now White is winning. The
way Steinitz concludes the game is highly spectacular; The
knight is coming to e6 with great effect; Now the g7 also is weak; [pgndiagram] A highly
attractive finish and one of the most remarkable combinations in the history
of chess.; A fantastic blow. At first sight it seems that
White will overextend himself, because his back rank is weak. But Steinitz has
calculted everything precisely; [pgndiagram] The best chance; The combination continues; Naturally, now Black can't
take the rook on g7 due to capture on d7 with check; 
[pgndiagram] Here Bardeleben left the game. Steinitz willingly demonstrated
the spectators what awaited his opponent had he continued the game;  }}}

\scalebox{0.90}{\chessboard}
\mainline{26. Rg7+ Kh8 27. Qh4+ Kxg7 28. Qh7+ Kf8 29. Qh8+ Ke7 30. Qg7+ Ke8 }
\xskakcomment{\small\texttt\justifying{\textcolor{darkgray}{~ 
[pgndiagram] Sharp gambits such as this one were quite popular in the 19th
century.; Not a critical reply; Now White has the initiative; [pgndiagram] A very counterintuitive
exchange. You can feel the understanding Steinitz had for the game from moves
like these. His games are full of deep strategic ideas.; The right
recapture; Once again the
right choice; 
[pgndiagram] If only Black could castle, he would have no problems whatsoever.
As it is, White has the initiative; [pgndiagram] So far the game
has been of a rather high quality. This move releases the tension; And immediately Bardeleben returns the favour; [pgndiagram] A thematic clearance move.Now White is winning. The
way Steinitz concludes the game is highly spectacular; The
knight is coming to e6 with great effect; Now the g7 also is weak; [pgndiagram] A highly
attractive finish and one of the most remarkable combinations in the history
of chess.; A fantastic blow. At first sight it seems that
White will overextend himself, because his back rank is weak. But Steinitz has
calculted everything precisely; [pgndiagram] The best chance; The combination continues; Naturally, now Black can't
take the rook on g7 due to capture on d7 with check; 
[pgndiagram] Here Bardeleben left the game. Steinitz willingly demonstrated
the spectators what awaited his opponent had he continued the game;  }}}

\scalebox{0.90}{\chessboard}
\mainline{31. Qg8+ Ke7 32. Qf7+ Kd8 33. Qf8+ Qe8 34. Nf7+ Kd7 35. Qd6# }
\xskakcomment{\small\texttt\justifying{\textcolor{darkgray}{~ 
[pgndiagram] Sharp gambits such as this one were quite popular in the 19th
century.; Not a critical reply; Now White has the initiative; [pgndiagram] A very counterintuitive
exchange. You can feel the understanding Steinitz had for the game from moves
like these. His games are full of deep strategic ideas.; The right
recapture; Once again the
right choice; 
[pgndiagram] If only Black could castle, he would have no problems whatsoever.
As it is, White has the initiative; [pgndiagram] So far the game
has been of a rather high quality. This move releases the tension; And immediately Bardeleben returns the favour; [pgndiagram] A thematic clearance move.Now White is winning. The
way Steinitz concludes the game is highly spectacular; The
knight is coming to e6 with great effect; Now the g7 also is weak; [pgndiagram] A highly
attractive finish and one of the most remarkable combinations in the history
of chess.; A fantastic blow. At first sight it seems that
White will overextend himself, because his back rank is weak. But Steinitz has
calculted everything precisely; [pgndiagram] The best chance; The combination continues; Naturally, now Black can't
take the rook on g7 due to capture on d7 with check; 
[pgndiagram] Here Bardeleben left the game. Steinitz willingly demonstrated
the spectators what awaited his opponent had he continued the game; [pgndiagram]
And at the end of the story, everything concludes with an epaulette mate.
Fantastic;  }}}

\scalebox{0.90}{\chessboard}

Score : 1-0

\end{multicols}

\newpage

\chessevent{Steinitz - Chigorin World Championship }
\chessopening{Havana CUB}

Date : 1892.01.07
EventDate : 1892.??.??
Round : 4
Result : 1-0
\whitename{Wilhelm Steinitz}
\blackname{Mikhail Chigorin}
\makegametitle
\begin{multicols}{2}
\noindent
\newchessgame[id=main]
\xskakset{style=styleC}
\mainline{1. e4 e5 2. Nf3 Nc6 3. Bb5 Nf6 4. d3 d6 5. c3 g6 }
\xskakcomment{\small\texttt\justifying{\textcolor{darkgray}{~ [pgndiagram] The way of avoiding Berlin
even in the 19th century.; The fianchetto against d3 and c3 is
encountered even today on an occasion.;  }}}

\scalebox{0.90}{\chessboard}
\mainline{6. Nbd2 Bg7 7. Nf1 O-O 8. Ba4 Nd7 9. Ne3 Nc5 10. Bc2 Ne6 }
\xskakcomment{\small\texttt\justifying{\textcolor{darkgray}{~ [pgndiagram] The way of avoiding Berlin
even in the 19th century.; The fianchetto against d3 and c3 is
encountered even today on an occasion.; But I am not a big
fan of this move. White violates one of the basic principles of not moving the
same piece twice in the opening.  Although the position is somewhat slow and
closed enough, I think that he loses some of his flexibility; the knight might
take the alternative Nc3-Ne3 route in the future if neccessary.; But this loses another tempo; [pgndiagram] Chigorin also moves his knight via c5
to e6, but this removes the knight from the kingside and loses time.; Chasing the bishop where
it wants to go;  }}}

\scalebox{0.90}{\chessboard}
\mainline{11. h4 Ne7 12. h5 d5 13. hxg6 fxg6 14. exd5 Nxd5 15. Nxd5 Qxd5 }
\xskakcomment{\small\texttt\justifying{\textcolor{darkgray}{~ [pgndiagram] The way of avoiding Berlin
even in the 19th century.; The fianchetto against d3 and c3 is
encountered even today on an occasion.; But I am not a big
fan of this move. White violates one of the basic principles of not moving the
same piece twice in the opening.  Although the position is somewhat slow and
closed enough, I think that he loses some of his flexibility; the knight might
take the alternative Nc3-Ne3 route in the future if neccessary.; But this loses another tempo; [pgndiagram] Chigorin also moves his knight via c5
to e6, but this removes the knight from the kingside and loses time.; Chasing the bishop where
it wants to go; 
[pgndiagram] Great understanding of the game by Steinitz. He recognizes that
Black is not in control in the centre, so he begins a wing attack; And
immediately there comes a mistake. Chigorin plays according to the principles
of the old school: attack on a flank should be countered by the centar blow.
However, White's centre is very stable here;  }}}

\scalebox{0.90}{\chessboard}

Score : 1-0

\end{multicols}

\newpage

\chessevent{London}
\chessopening{London}

Date : 1862.??.??
EventDate : 1862.??.??
Round : ?
Result : 1-0
\whitename{Wilhelm Steinitz}
\blackname{Augustus Mongredien}
\makegametitle
\begin{multicols}{2}
\noindent
\newchessgame[id=main]
\xskakset{style=styleC}
\mainline{1. e4 d5 2. exd5 Qxd5 3. Nc3 Qd8 4. d4 e6 5. Nf3 Nf6 }
\xskakcomment{\small\texttt\justifying{\textcolor{darkgray}{~ [pgndiagram] It is interesting to see
this line of Scandinavian in a game from 1862.; I am not an expert
on this variation, but I believe that Black should consider playing some other
moves and wait for an opportunity to pin the knight on f3. Here his light
squared bishop will remain a problem piece;  }}}

\scalebox{0.90}{\chessboard}
\mainline{6. Bd3 Be7 7. O-O O-O 8. Be3 b6 9. Ne5 Bb7 10. f4 Nbd7 }
\xskakcomment{\small\texttt\justifying{\textcolor{darkgray}{~ [pgndiagram] It is interesting to see
this line of Scandinavian in a game from 1862.; I am not an expert
on this variation, but I believe that Black should consider playing some other
moves and wait for an opportunity to pin the knight on f3. Here his light
squared bishop will remain a problem piece; [pgndiagram] I am not certain
about this move. It does develop a piece, but the bishop on e3 is not optimal.; Steinitz goes all in, but Black
could have defended;  }}}

\scalebox{0.90}{\chessboard}
\mainline{11. Qe2 Nd5 12. Nxd5 exd5 13. Rf3 f5 14. Rh3 g6 15. g4 fxg4 }
\xskakcomment{\small\texttt\justifying{\textcolor{darkgray}{~ [pgndiagram] It is interesting to see
this line of Scandinavian in a game from 1862.; I am not an expert
on this variation, but I believe that Black should consider playing some other
moves and wait for an opportunity to pin the knight on f3. Here his light
squared bishop will remain a problem piece; [pgndiagram] I am not certain
about this move. It does develop a piece, but the bishop on e3 is not optimal.; Steinitz goes all in, but Black
could have defended; Now White's pieces enjoy excellent prospects; Not a
move made willingly; Black has to prevent the move Qh5; Going
for it. The computer doesn't like this move, but for a human this is very
attractive; [pgndiagram] And immediately there comes a big mistake,
which loses;  }}}

\scalebox{0.90}{\chessboard}

Score : 1-0

\end{multicols}

\newpage

\chessevent{London (England)}
\chessopening{London (England)}

Date : 1863.??.??
EventDate : 1863.??.??
Round : ?
Result : 1-0
\whitename{Wilhelm Steinitz}
\blackname{Augustus Mongredien}
\makegametitle
\begin{multicols}{2}
\noindent
\newchessgame[id=main]
\xskakset{style=styleC}
\mainline{1. e4 g6 2. d4 Bg7 3. c3 b6 4. Be3 Bb7 5. Nd2 d6 }
\xskakcomment{\small\texttt\justifying{\textcolor{darkgray}{~ I have chosen this game only because it featured the modern defence; This move is death for every Modern player; And with
this we have entered the uncharted waters.; I think that
combination of Be3 and c3 is not good. The d4 square is safe enough. On e3
bishop might become a subject of the Ng4 in the future;  }}}

\scalebox{0.90}{\chessboard}
\mainline{6. Ngf3 e5 7. dxe5 dxe5 8. Bc4 Ne7 9. Qe2 O-O 10. h4 Nd7 }
\xskakcomment{\small\texttt\justifying{\textcolor{darkgray}{~ I have chosen this game only because it featured the modern defence; This move is death for every Modern player; And with
this we have entered the uncharted waters.; I think that
combination of Be3 and c3 is not good. The d4 square is safe enough. On e3
bishop might become a subject of the Ng4 in the future; [pgndiagram] Generally speaking, the combination of the queenside
fianchetto together with the e5 advance should be condemned.; Now Black can't play Nf6 and has some tactical problems as well; The losing move; another mistake. Castling
without the knight on f6 and with the rook on h1 is very dangerous; [pgndiagram] A
familiar advance of Harry the h-pawn;  }}}

\scalebox{0.90}{\chessboard}
\mainline{11. h5 Nf6 12. hxg6 Nxg6 13. O-O-O c5 14. Ng5 a6 15. Nxh7 Nxh7 }
\xskakcomment{\small\texttt\justifying{\textcolor{darkgray}{~ I have chosen this game only because it featured the modern defence; This move is death for every Modern player; And with
this we have entered the uncharted waters.; I think that
combination of Be3 and c3 is not good. The d4 square is safe enough. On e3
bishop might become a subject of the Ng4 in the future; [pgndiagram] Generally speaking, the combination of the queenside
fianchetto together with the e5 advance should be condemned.; Now Black can't play Nf6 and has some tactical problems as well; The losing move; another mistake. Castling
without the knight on f6 and with the rook on h1 is very dangerous; [pgndiagram] A
familiar advance of Harry the h-pawn; White has a
decisive attack; Defending the e-pawn; A final mistake of the game; [pgndiagram] Another sacrifice on h7 and another correct one;  }}}

\scalebox{0.90}{\chessboard}

Score : 1-0

\end{multicols}

\newpage

\chessevent{Anderssen - Steinitz}
\chessopening{London ENG}

Date : 1866.08.08
EventDate : 1866.07.18
Round : 13
Result : 0-1
\whitename{Adolf Anderssen}
\blackname{Wilhelm Steinitz}
\makegametitle
\begin{multicols}{2}
\noindent
\newchessgame[id=main]
\xskakset{style=styleC}
\mainline{1. e4 e5 2. Nf3 Nc6 3. Bb5 Nf6 4. d3 d6 5. Bxc6+ bxc6 }
\scalebox{0.90}{\chessboard}
\mainline{6. h3 g6 7. Nc3 Bg7 8. O-O O-O 9. Bg5 h6 10. Be3 c5 }
\xskakcomment{\small\texttt\justifying{\textcolor{darkgray}{~ [pgndiagram]
These Anti-Berlin lines with kingside fianchetto were championed successfully
by Steinitz;  }}}

\scalebox{0.90}{\chessboard}
\mainline{11. Rb1 Ne8 12. b4 cxb4 13. Rxb4 c5 14. Ra4 Bd7 15. Ra3 f5 }
\xskakcomment{\small\texttt\justifying{\textcolor{darkgray}{~ [pgndiagram]
These Anti-Berlin lines with kingside fianchetto were championed successfully
by Steinitz; 
[pgndiagram] A very modern interpretation of the opening for those years.
White recognizes that he has to play on the queenside and decides to undouble
the adversary pawns in order to open a file for his rook; Only this is slightly misguided. White removes his rook from the open
file;  }}}

\scalebox{0.90}{\chessboard}
\mainline{16. Qb1 Kh8 17. Qb7 a5 18. Rb1 a4 19. Qd5 Qc8 20. Rb6 Ra7 }
\xskakcomment{\small\texttt\justifying{\textcolor{darkgray}{~ [pgndiagram]
These Anti-Berlin lines with kingside fianchetto were championed successfully
by Steinitz; 
[pgndiagram] A very modern interpretation of the opening for those years.
White recognizes that he has to play on the queenside and decides to undouble
the adversary pawns in order to open a file for his rook; Only this is slightly misguided. White removes his rook from the open
file; [pgndiagram] From this moment onwards,
White's game starts going downhill rapidly. His pieces are stuck on the
queenside and are unable to join the kingside and defend the king;  }}}

\scalebox{0.90}{\chessboard}
\mainline{21. Kh2 f4 22. Bd2 g5 23. Qc4 Qd8 24. Rb1 Nf6 25. Kg1 Nh7 }
\xskakcomment{\small\texttt\justifying{\textcolor{darkgray}{~ [pgndiagram]
These Anti-Berlin lines with kingside fianchetto were championed successfully
by Steinitz; 
[pgndiagram] A very modern interpretation of the opening for those years.
White recognizes that he has to play on the queenside and decides to undouble
the adversary pawns in order to open a file for his rook; Only this is slightly misguided. White removes his rook from the open
file; [pgndiagram] From this moment onwards,
White's game starts going downhill rapidly. His pieces are stuck on the
queenside and are unable to join the kingside and defend the king; [pgndiagram] Black finally decides
on this move. He will soon set his pawn chain in motion, and alraedy it is
very hard to suggest how White should counter that plan; Defending g5 and intending h5; Going from frying pan to fire; But anyway, White's position is unpleasant; h5-g4 is coming soon;  }}}

\scalebox{0.90}{\chessboard}

Score : 0-1

\end{multicols}

\newpage

\chessevent{Steinitz - Blackburne}
\chessopening{London ENG}

Date : 1876.02.29
EventDate : 1876.02.17
Round : 6
Result : 0-1
\whitename{Joseph Henry Blackburne}
\blackname{Wilhelm Steinitz}
\makegametitle
\begin{multicols}{2}
\noindent
\newchessgame[id=main]
\xskakset{style=styleC}
\mainline{1. e4 e5 2. Nf3 Nc6 3. d4 exd4 4. Nxd4 Qh4 5. Nb5 Bb4+ }
\xskakcomment{\small\texttt\justifying{\textcolor{darkgray}{~ [pgndiagram] This variation of the
Scotch defence is nowadays considered dangerous. But Steinitz simply grabs the
pawn and defends it for the remainder of the game.;  }}}

\scalebox{0.90}{\chessboard}
\mainline{6. Bd2 Qxe4+ 7. Be2 Kd8 8. O-O Bxd2 9. Qxd2 a6 10. N5a3 Qd4 }
\xskakcomment{\small\texttt\justifying{\textcolor{darkgray}{~ [pgndiagram] This variation of the
Scotch defence is nowadays considered dangerous. But Steinitz simply grabs the
pawn and defends it for the remainder of the game.; [pgndiagram] Black has given up a pawn for the right to
castle. Objectively speaking, White should have decent compensation for the
pawn, but it is not so easy to prove it.; Now Black has time to chase the knight away from d4;  }}}

\scalebox{0.90}{\chessboard}
\mainline{11. Qg5+ Qf6 12. Qd2 Qxb2 13. Nc4 Qd4 14. Qc1 Nge7 15. Nbd2 d6 }
\xskakcomment{\small\texttt\justifying{\textcolor{darkgray}{~ [pgndiagram] This variation of the
Scotch defence is nowadays considered dangerous. But Steinitz simply grabs the
pawn and defends it for the remainder of the game.; [pgndiagram] Black has given up a pawn for the right to
castle. Objectively speaking, White should have decent compensation for the
pawn, but it is not so easy to prove it.; Now Black has time to chase the knight away from d4; [pgndiagram] Giving another pawn away is
too much; 
[pgndiagram] Even so, Black has retained the advantage. Two pawns are too much;  }}}

\scalebox{0.90}{\chessboard}
\mainline{16. Rd1 Be6 17. Qa3 Nd5 18. Nb3 Qc3 19. Bf1 Ndb4 20. Ne3 Re8 }
\xskakcomment{\small\texttt\justifying{\textcolor{darkgray}{~ [pgndiagram] This variation of the
Scotch defence is nowadays considered dangerous. But Steinitz simply grabs the
pawn and defends it for the remainder of the game.; [pgndiagram] Black has given up a pawn for the right to
castle. Objectively speaking, White should have decent compensation for the
pawn, but it is not so easy to prove it.; Now Black has time to chase the knight away from d4; [pgndiagram] Giving another pawn away is
too much; 
[pgndiagram] Even so, Black has retained the advantage. Two pawns are too much; 
[pgndiagram]Black has consolidated completely and now only needs to activate
his rooks;  }}}

\scalebox{0.90}{\chessboard}
\mainline{21. Rd2 Bxb3 22. Rad1 Rxe3 23. fxe3 Nxc2 24. Qc1 Qxe3+ 25. Kh1 Ba4 }
\xskakcomment{\small\texttt\justifying{\textcolor{darkgray}{~ [pgndiagram] This variation of the
Scotch defence is nowadays considered dangerous. But Steinitz simply grabs the
pawn and defends it for the remainder of the game.; [pgndiagram] Black has given up a pawn for the right to
castle. Objectively speaking, White should have decent compensation for the
pawn, but it is not so easy to prove it.; Now Black has time to chase the knight away from d4; [pgndiagram] Giving another pawn away is
too much; 
[pgndiagram] Even so, Black has retained the advantage. Two pawns are too much; 
[pgndiagram]Black has consolidated completely and now only needs to activate
his rooks; Desperation. White embarks on a faulty
combination; [pgndiagram] A nice tactical refutation; [pgndiagram] Now Black is
easily winning (a piece and four pawns for the rook).;  }}}

\scalebox{0.90}{\chessboard}
\mainline{26. Bc4 N2d4 27. Re1 Qf4 28. Rf1 Qh6 29. Qb2 Qe3 30. Bxf7 Bb5 }
\xskakcomment{\small\texttt\justifying{\textcolor{darkgray}{~ [pgndiagram] This variation of the
Scotch defence is nowadays considered dangerous. But Steinitz simply grabs the
pawn and defends it for the remainder of the game.; [pgndiagram] Black has given up a pawn for the right to
castle. Objectively speaking, White should have decent compensation for the
pawn, but it is not so easy to prove it.; Now Black has time to chase the knight away from d4; [pgndiagram] Giving another pawn away is
too much; 
[pgndiagram] Even so, Black has retained the advantage. Two pawns are too much; 
[pgndiagram]Black has consolidated completely and now only needs to activate
his rooks; Desperation. White embarks on a faulty
combination; [pgndiagram] A nice tactical refutation; [pgndiagram] Now Black is
easily winning (a piece and four pawns for the rook).;  }}}

\scalebox{0.90}{\chessboard}
\mainline{31. Rfd1 Nf5 32. a4 Ne5 33. axb5 Nxf7 34. Re2 Qh6 35. Qb3 axb5 }
\xskakcomment{\small\texttt\justifying{\textcolor{darkgray}{~ [pgndiagram] This variation of the
Scotch defence is nowadays considered dangerous. But Steinitz simply grabs the
pawn and defends it for the remainder of the game.; [pgndiagram] Black has given up a pawn for the right to
castle. Objectively speaking, White should have decent compensation for the
pawn, but it is not so easy to prove it.; Now Black has time to chase the knight away from d4; [pgndiagram] Giving another pawn away is
too much; 
[pgndiagram] Even so, Black has retained the advantage. Two pawns are too much; 
[pgndiagram]Black has consolidated completely and now only needs to activate
his rooks; Desperation. White embarks on a faulty
combination; [pgndiagram] A nice tactical refutation; [pgndiagram] Now Black is
easily winning (a piece and four pawns for the rook).;  }}}

\scalebox{0.90}{\chessboard}

Score : 0-1

\end{multicols}

\newpage

\chessevent{Vienna}
\chessopening{Vienna AUT}

Date : 1873.08.12
EventDate : 1873.07.21
Round : 7.2
Result : 0-1
\whitename{Adolf Anderssen}
\blackname{Wilhelm Steinitz}
\makegametitle
\begin{multicols}{2}
\noindent
\newchessgame[id=main]
\xskakset{style=styleC}
\mainline{1. e4 e5 2. Nf3 Nc6 3. Bb5 a6 4. Ba4 Nf6 5. d3 d6 }
\xskakcomment{\small\texttt\justifying{\textcolor{darkgray}{~ One of the rare instances of Steinitz starting
with 3.. a6 instead of 3... Nf6 or 3... d6; But still,
Anderssen chooses this move, leading to an Anti-Berlin, like position.;  }}}

\scalebox{0.90}{\chessboard}
\mainline{6. Bxc6+ bxc6 7. h3 g6 8. Nc3 Bg7 9. Be3 Rb8 10. b3 c5 }
\xskakcomment{\small\texttt\justifying{\textcolor{darkgray}{~ One of the rare instances of Steinitz starting
with 3.. a6 instead of 3... Nf6 or 3... d6; But still,
Anderssen chooses this move, leading to an Anti-Berlin, like position.; [pgndiagram] And again we have the fianchetto and the
exhcange on c6, similarly to the previously examined encounter; This time Steinitz doesn't go for c5 but seizes the b-file himself;  }}}

\scalebox{0.90}{\chessboard}
\mainline{11. Qd2 h6 12. g4 Ng8 13. O-O-O Ne7 14. Ne2 Nc6 15. Qc3 Nd4 }
\xskakcomment{\small\texttt\justifying{\textcolor{darkgray}{~ One of the rare instances of Steinitz starting
with 3.. a6 instead of 3... Nf6 or 3... d6; But still,
Anderssen chooses this move, leading to an Anti-Berlin, like position.; [pgndiagram] And again we have the fianchetto and the
exhcange on c6, similarly to the previously examined encounter; This time Steinitz doesn't go for c5 but seizes the b-file himself; It is doubtful whether this move is necessary.; [pgndiagram] But this is completely uncalled for; Very strong play by Steinitz. He
starts moving his knight toward d4 immediately; This knight is going nowhere.
Although the position is mildly unpleasant for White, notice how quickly
Steinitz will outplay his opponent; The
queen is completely missplaced here;  }}}

\scalebox{0.90}{\chessboard}
\mainline{16. Nfg1 O-O 17. Ng3 Be6 18. N1e2 Qd7 19. Bxd4 cxd4 20. Qb2 a5 }
\xskakcomment{\small\texttt\justifying{\textcolor{darkgray}{~ One of the rare instances of Steinitz starting
with 3.. a6 instead of 3... Nf6 or 3... d6; But still,
Anderssen chooses this move, leading to an Anti-Berlin, like position.; [pgndiagram] And again we have the fianchetto and the
exhcange on c6, similarly to the previously examined encounter; This time Steinitz doesn't go for c5 but seizes the b-file himself; It is doubtful whether this move is necessary.; [pgndiagram] But this is completely uncalled for; Very strong play by Steinitz. He
starts moving his knight toward d4 immediately; This knight is going nowhere.
Although the position is mildly unpleasant for White, notice how quickly
Steinitz will outplay his opponent; The
queen is completely missplaced here; [pgndiagram] Black has once again gained a strong
attack, while White's counterintuitive on the other wing is non existent;  }}}

\scalebox{0.90}{\chessboard}
\mainline{21. Kd2 d5 22. f3 Qe7 23. Rdf1 Qb4+ 24. Kd1 a4 25. Rh2 c5 }
\xskakcomment{\small\texttt\justifying{\textcolor{darkgray}{~ One of the rare instances of Steinitz starting
with 3.. a6 instead of 3... Nf6 or 3... d6; But still,
Anderssen chooses this move, leading to an Anti-Berlin, like position.; [pgndiagram] And again we have the fianchetto and the
exhcange on c6, similarly to the previously examined encounter; This time Steinitz doesn't go for c5 but seizes the b-file himself; It is doubtful whether this move is necessary.; [pgndiagram] But this is completely uncalled for; Very strong play by Steinitz. He
starts moving his knight toward d4 immediately; This knight is going nowhere.
Although the position is mildly unpleasant for White, notice how quickly
Steinitz will outplay his opponent; The
queen is completely missplaced here; [pgndiagram] Black has once again gained a strong
attack, while White's counterintuitive on the other wing is non existent;  }}}

\scalebox{0.90}{\chessboard}
\mainline{26. Nc1 c4 27. a3 Qe7 28. b4 c3 29. Qa1 Qg5 30. Rff2 f5 }
\xskakcomment{\small\texttt\justifying{\textcolor{darkgray}{~ One of the rare instances of Steinitz starting
with 3.. a6 instead of 3... Nf6 or 3... d6; But still,
Anderssen chooses this move, leading to an Anti-Berlin, like position.; [pgndiagram] And again we have the fianchetto and the
exhcange on c6, similarly to the previously examined encounter; This time Steinitz doesn't go for c5 but seizes the b-file himself; It is doubtful whether this move is necessary.; [pgndiagram] But this is completely uncalled for; Very strong play by Steinitz. He
starts moving his knight toward d4 immediately; This knight is going nowhere.
Although the position is mildly unpleasant for White, notice how quickly
Steinitz will outplay his opponent; The
queen is completely missplaced here; [pgndiagram] Black has once again gained a strong
attack, while White's counterintuitive on the other wing is non existent; [pgndiagram] White's queen is
completely sidelined and Black obtains free hands on the kingside. It is very
instructive to watch how Steinitz uses his greater control of space to quickly
shift his attention to the other side of the board;  }}}

\scalebox{0.90}{\chessboard}
\mainline{31. exf5 gxf5 32. h4 Qg6 33. Nxf5 Bxf5 34. gxf5 Rxf5 35. Ne2 Rbf8 }
\xskakcomment{\small\texttt\justifying{\textcolor{darkgray}{~ One of the rare instances of Steinitz starting
with 3.. a6 instead of 3... Nf6 or 3... d6; But still,
Anderssen chooses this move, leading to an Anti-Berlin, like position.; [pgndiagram] And again we have the fianchetto and the
exhcange on c6, similarly to the previously examined encounter; This time Steinitz doesn't go for c5 but seizes the b-file himself; It is doubtful whether this move is necessary.; [pgndiagram] But this is completely uncalled for; Very strong play by Steinitz. He
starts moving his knight toward d4 immediately; This knight is going nowhere.
Although the position is mildly unpleasant for White, notice how quickly
Steinitz will outplay his opponent; The
queen is completely missplaced here; [pgndiagram] Black has once again gained a strong
attack, while White's counterintuitive on the other wing is non existent; [pgndiagram] White's queen is
completely sidelined and Black obtains free hands on the kingside. It is very
instructive to watch how Steinitz uses his greater control of space to quickly
shift his attention to the other side of the board; [pgndiagram] Opening the kingside. White is completely lost; Losing immediately;  }}}

\scalebox{0.90}{\chessboard}
\mainline{36. Qa2 Qf7 37. Rh3 Kh7 38. Ng1 Bf6 39. Ke2 Rg8 40. Kf1 Be7 }
\xskakcomment{\small\texttt\justifying{\textcolor{darkgray}{~ One of the rare instances of Steinitz starting
with 3.. a6 instead of 3... Nf6 or 3... d6; But still,
Anderssen chooses this move, leading to an Anti-Berlin, like position.; [pgndiagram] And again we have the fianchetto and the
exhcange on c6, similarly to the previously examined encounter; This time Steinitz doesn't go for c5 but seizes the b-file himself; It is doubtful whether this move is necessary.; [pgndiagram] But this is completely uncalled for; Very strong play by Steinitz. He
starts moving his knight toward d4 immediately; This knight is going nowhere.
Although the position is mildly unpleasant for White, notice how quickly
Steinitz will outplay his opponent; The
queen is completely missplaced here; [pgndiagram] Black has once again gained a strong
attack, while White's counterintuitive on the other wing is non existent; [pgndiagram] White's queen is
completely sidelined and Black obtains free hands on the kingside. It is very
instructive to watch how Steinitz uses his greater control of space to quickly
shift his attention to the other side of the board; [pgndiagram] Opening the kingside. White is completely lost; Losing immediately;  }}}

\scalebox{0.90}{\chessboard}
\mainline{41. Ne2 Rh5 42. f4 Bxh4 43. Rff3 e4 44. dxe4 Qg6 45. Ng3 }
\xskakcomment{\small\texttt\justifying{\textcolor{darkgray}{~ One of the rare instances of Steinitz starting
with 3.. a6 instead of 3... Nf6 or 3... d6; But still,
Anderssen chooses this move, leading to an Anti-Berlin, like position.; [pgndiagram] And again we have the fianchetto and the
exhcange on c6, similarly to the previously examined encounter; This time Steinitz doesn't go for c5 but seizes the b-file himself; It is doubtful whether this move is necessary.; [pgndiagram] But this is completely uncalled for; Very strong play by Steinitz. He
starts moving his knight toward d4 immediately; This knight is going nowhere.
Although the position is mildly unpleasant for White, notice how quickly
Steinitz will outplay his opponent; The
queen is completely missplaced here; [pgndiagram] Black has once again gained a strong
attack, while White's counterintuitive on the other wing is non existent; [pgndiagram] White's queen is
completely sidelined and Black obtains free hands on the kingside. It is very
instructive to watch how Steinitz uses his greater control of space to quickly
shift his attention to the other side of the board; [pgndiagram] Opening the kingside. White is completely lost; Losing immediately;  }}}

\scalebox{0.90}{\chessboard}

Score : 0-1

\end{multicols}

\newpage

\chessevent{Steinitz - Zukertort World Championship}
\chessopening{New York, NY USA}

Date : 1886.01.11
EventDate : 1886.??.??
Round : 1
Result : 0-1
\whitename{Johannes Zukertort}
\blackname{Wilhelm Steinitz}
\makegametitle
\begin{multicols}{2}
\noindent
\newchessgame[id=main]
\xskakset{style=styleC}
\mainline{1. d4 d5 2. c4 c6 3. e3 Bf5 4. Nc3 e6 5. Nf3 Nd7 }
\xskakcomment{\small\texttt\justifying{\textcolor{darkgray}{~ It was rare to see queen's pawn moving on the first move in those times; The Slav! In 1886!;  }}}

\scalebox{0.90}{\chessboard}
\mainline{6. a3 Bd6 7. c5 Bc7 8. b4 e5 9. Be2 Ngf6 10. Bb2 e4 }
\xskakcomment{\small\texttt\justifying{\textcolor{darkgray}{~ It was rare to see queen's pawn moving on the first move in those times; The Slav! In 1886!; A
sign of the times. Obviously there is a lot of theory and lot of possibilities.
I am sure I could write a good book on all the nuances and move orders here.; Provoking White's next move; Tempting but faulty. Black will
be able to break the bind by the timely e5 move; Thematic in similar structures. White has a
bind on the queenside; Black has to break in the centre;  }}}

\scalebox{0.90}{\chessboard}
\mainline{11. Nd2 h5 12. h3 Nf8 13. a4 Ng6 14. b5 Nh4 15. g3 Ng2+ }
\xskakcomment{\small\texttt\justifying{\textcolor{darkgray}{~ It was rare to see queen's pawn moving on the first move in those times; The Slav! In 1886!; A
sign of the times. Obviously there is a lot of theory and lot of possibilities.
I am sure I could write a good book on all the nuances and move orders here.; Provoking White's next move; Tempting but faulty. Black will
be able to break the bind by the timely e5 move; Thematic in similar structures. White has a
bind on the queenside; Black has to break in the centre; 
Steinitz once again plays principled chess. He gains space on the kingside and
intends to mobilize his every single piece on that area of the board; [pgndiagram] To my mind this is a
fantastic move. Had such a conception been played in the 21st century,
everyone would assume it was computer preparation. However, it was 1886... 
Steinitz embarks on a modern knight sacrifice, for which he gains long term
compensation;  }}}

\scalebox{0.90}{\chessboard}
\mainline{16. Kf1 Nxe3+ 17. fxe3 Bxg3 18. Kg2 Bc7 19. Qg1 Rh6 20. Kf1 Rg6 }
\xskakcomment{\small\texttt\justifying{\textcolor{darkgray}{~ It was rare to see queen's pawn moving on the first move in those times; The Slav! In 1886!; A
sign of the times. Obviously there is a lot of theory and lot of possibilities.
I am sure I could write a good book on all the nuances and move orders here.; Provoking White's next move; Tempting but faulty. Black will
be able to break the bind by the timely e5 move; Thematic in similar structures. White has a
bind on the queenside; Black has to break in the centre; 
Steinitz once again plays principled chess. He gains space on the kingside and
intends to mobilize his every single piece on that area of the board; [pgndiagram] To my mind this is a
fantastic move. Had such a conception been played in the 21st century,
everyone would assume it was computer preparation. However, it was 1886... 
Steinitz embarks on a modern knight sacrifice, for which he gains long term
compensation; And here we go. Once again we see the concept
of space advantage utilized. White's pieces, stuck on the queenside will be
unable to come and defend White's king, who is stuck on the kingside; [pgndiagram] This is probably faulty retreat, as it gives
White enough time to organize; Returning the favour;  }}}

\scalebox{0.90}{\chessboard}
\mainline{21. Qf2 Qd7 22. bxc6 bxc6 23. Rg1 Bxh3+ 24. Ke1 Ng4 25. Bxg4 Bxg4 }
\xskakcomment{\small\texttt\justifying{\textcolor{darkgray}{~ It was rare to see queen's pawn moving on the first move in those times; The Slav! In 1886!; A
sign of the times. Obviously there is a lot of theory and lot of possibilities.
I am sure I could write a good book on all the nuances and move orders here.; Provoking White's next move; Tempting but faulty. Black will
be able to break the bind by the timely e5 move; Thematic in similar structures. White has a
bind on the queenside; Black has to break in the centre; 
Steinitz once again plays principled chess. He gains space on the kingside and
intends to mobilize his every single piece on that area of the board; [pgndiagram] To my mind this is a
fantastic move. Had such a conception been played in the 21st century,
everyone would assume it was computer preparation. However, it was 1886... 
Steinitz embarks on a modern knight sacrifice, for which he gains long term
compensation; And here we go. Once again we see the concept
of space advantage utilized. White's pieces, stuck on the queenside will be
unable to come and defend White's king, who is stuck on the kingside; [pgndiagram] This is probably faulty retreat, as it gives
White enough time to organize; Returning the favour; [pgndiagram] Now White
doesn't have enough time. h3 pawn falls as well; 
White has to part with the h3 pawn; [pgndiagram] And just like that, Black is simply winning. Bear in
mind that Zuketort was the strongest player apart from Steinitz at that moment.
;  }}}

\scalebox{0.90}{\chessboard}

Score : 0-1

\end{multicols}

\newpage

\chessevent{Steinitz - Zukertort World Championship}
\chessopening{New Orleans, LA USA}

Date : 1886.03.24
EventDate : 1886.??.??
Round : 19
Result : 0-1
\whitename{Johannes Zukertort}
\blackname{Wilhelm Steinitz}
\makegametitle
\begin{multicols}{2}
\noindent
\newchessgame[id=main]
\xskakset{style=styleC}
\mainline{1. d4 d5 2. c4 e6 3. Nc3 Nf6 4. Bg5 Be7 5. Nf3 O-O }
\xskakcomment{\small\texttt\justifying{\textcolor{darkgray}{~ [pgndiagram] Steinitz deviates from the Slav
which had brought him success earlier in the match;  }}}

\scalebox{0.90}{\chessboard}
\mainline{6. c5 b6 7. b4 bxc5 8. dxc5 a5 9. a3 d4 10. Bxf6 gxf6 }
\xskakcomment{\small\texttt\justifying{\textcolor{darkgray}{~ [pgndiagram] Steinitz deviates from the Slav
which had brought him success earlier in the match; [pgndiagram] The best proof that queen's pawn openings were terra
incognita for Zukertort. This move gives Black immediate targets.; Returning the favour; The errors
continue; [pgndiagram] Black's position is so
good, that virtually everything works for him.;  }}}

\scalebox{0.90}{\chessboard}
\mainline{11. Na4 e5 12. b5 Be6 13. g3 c6 14. bxc6 Nxc6 15. Bg2 Rb8 }
\xskakcomment{\small\texttt\justifying{\textcolor{darkgray}{~ [pgndiagram] Steinitz deviates from the Slav
which had brought him success earlier in the match; [pgndiagram] The best proof that queen's pawn openings were terra
incognita for Zukertort. This move gives Black immediate targets.; Returning the favour; The errors
continue; [pgndiagram] Black's position is so
good, that virtually everything works for him.; Now White manages to advance his
pawns and is probably not even worse; Probably it was wiser not
to allow c6; [pgndiagram] But
now once again White's position is highly unpleasant; Threatening
Bb3 and not allowing White to castle;  }}}

\scalebox{0.90}{\chessboard}

Score : 0-1

\end{multicols}

\newpage

\chessevent{Steinitz - Zukertort World Championship}
\chessopening{New Orleans, LA USA}

Date : 1886.03.29
EventDate : 1886.??.??
Round : 20
Result : 1-0
\whitename{Wilhelm Steinitz}
\blackname{Johannes Zukertort}
\makegametitle
\begin{multicols}{2}
\noindent
\newchessgame[id=main]
\xskakset{style=styleC}
\mainline{1. e4 e5 2. Nc3 Nc6 3. f4 exf4 4. d4 d5 5. exd5 Qh4+ }
\xskakcomment{\small\texttt\justifying{\textcolor{darkgray}{~ Steinitz has developed this audacious
gambit in his youth. It is interesting to see playing it in the most important
moment of his career - the World Championship match.; Playing d5 before
going for the check is worse, since White can take on f4 now as well; Allowing the
check, naturally;  }}}

\scalebox{0.90}{\chessboard}
\mainline{6. Ke2 Qe7+ 7. Kf2 Qh4+ 8. g3 fxg3+ 9. Kg2 Nxd4 10. hxg3 Qg4 }
\xskakcomment{\small\texttt\justifying{\textcolor{darkgray}{~ Steinitz has developed this audacious
gambit in his youth. It is interesting to see playing it in the most important
moment of his career - the World Championship match.; Playing d5 before
going for the check is worse, since White can take on f4 now as well; Allowing the
check, naturally; [pgndiagram] Actually, this way of
refuting the gambit is probably not the most energetic and Black has to look
for other possibilities instead of 4 (5).. d5;  }}}

\scalebox{0.90}{\chessboard}

Score : 1-0

\end{multicols}

\newpage

\end{document}
