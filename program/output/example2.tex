\documentclass{article}
\usepackage{multicol}
\usepackage{array}
\usepackage{makeidx}
\usepackage[skaknew]{chessfss}
\usepackage{texmate}
\usepackage{xskak}
\usepackage[top=1.5cm, bottom=2cm, left=1.5cm, right=1cm,headheight=15pt]{geometry}
\usepackage{adjmulticol}
\usepackage{ragged2e}


\begin{document}

\chessevent{Hastings}
\chessopening{Hastings ENG}

Date : 1895.08.17

Round : 10

Result : 1-0

\whitename{Wilhelm Steinitz}

\blackname{Curt von Bardeleben}

\ECO{C54}

\whiteelo{?}

\blackelo{?}

Plycount : 69

\makegametitle
\begin{multicols}{2}
\noindent
\newchessgame[id=main]
\xskakset{style=styleC}
\mainline{1. e4 e5 2. Nf3 Nc6 }
\scalebox{0.90}{\chessboard}\\
\mainline{3. Bc4 Bc5 4. c3 Nf6 }
\scalebox{0.90}{\chessboard}\\
\mainline{5. d4 exd4 6. cxd4 Bb4+ }
\scalebox{0.90}{\chessboard}\\
\mainline{7. Nc3 }
\xskakcomment{\small\texttt\justifying{\textcolor{darkgray}{
[pgndiagram] Sharp gambits such as this one were quite popular in the 19th
century.}}}
\mainline{7... d5 }
\xskakcomment{\small\texttt\justifying{\textcolor{darkgray}{Not a critical reply}}}

\xskakcomment{\small\texttt\justifying{\textcolor{gray}{Variant : \variation{7... Nxe4} "The best way of refuting a
gambit is to accept it" }}}
\mainline{8. exd5 }
\xskakcomment{\small\texttt\justifying{\textcolor{darkgray}{Now White has the initiative}}}
\mainline{8... Nxd5 }
\scalebox{0.90}{\chessboard}\\
\mainline{9. O-O Be6 }
\xskakcomment{\small\texttt\justifying{\textcolor{gray}{Variant n°1: \variation{9... Bxc3} 
Here Black can't go for the pawn }}}

\xskakcomment{\small\texttt\justifying{\textcolor{gray}{Variant n°2: \variation{9... Nxc3} Again loses to }}}
\mainline{10. Bg5 Be7 }
\xskakcomment{\small\texttt\justifying{\textcolor{gray}{Variant : \variation{10... Qd7} Was arguably
better }}}
\mainline{}
\scalebox{0.90}{\chessboard}\\
\mainline{11. Bxd5 }
\xskakcomment{\small\texttt\justifying{\textcolor{darkgray}{[pgndiagram] A very counterintuitive
exchange. You can feel the understanding Steinitz had for the game from moves
like these. His games are full of deep strategic ideas.}}}
\mainline{11... Bxd5 }
\xskakcomment{\small\texttt\justifying{\textcolor{darkgray}{The right
recapture}}}

\xskakcomment{\small\texttt\justifying{\textcolor{gray}{Variant : \variation{11... Bxg5} Is horrible for Black }}}
\mainline{12. Nxd5 Qxd5 }
\xskakcomment{\small\texttt\justifying{\textcolor{darkgray}{Once again the
right choice}}}

\xskakcomment{\small\texttt\justifying{\textcolor{gray}{Variant : \variation{12... Bxg5} Is bad due to }}}
\mainline{}
\scalebox{0.90}{\chessboard}\\
\mainline{13. Bxe7 Nxe7 14. Re1 }
\xskakcomment{\small\texttt\justifying{\textcolor{darkgray}{
[pgndiagram] If only Black could castle, he would have no problems whatsoever.
As it is, White has the initiative}}}
\mainline{14... f6 }
\xskakcomment{\small\texttt\justifying{\textcolor{gray}{Variant n°1: \variation{14... Qd7} Would allow the extra tempo
on the knight }}}

\xskakcomment{\small\texttt\justifying{\textcolor{gray}{Variant n°2: \variation{14... Qd6} Would
likely transpose, although the queen on d6 is subjecto to knight attack from
e4 or c4 }}}
\mainline{}
\scalebox{0.90}{\chessboard}\\
\mainline{15. Qe2 Qd7 16. Rac1 }
\xskakcomment{\small\texttt\justifying{\textcolor{darkgray}{[pgndiagram] So far the game
has been of a rather high quality. This move releases the tension}}}

\xskakcomment{\small\texttt\justifying{\textcolor{gray}{Variant : \variation{16. d5} }}}
\mainline{16... c6 }
\xskakcomment{\small\texttt\justifying{\textcolor{darkgray}{And immediately Bardeleben returns the favour}}}

\xskakcomment{\small\texttt\justifying{\textcolor{gray}{Variant : \variation{16... Kf7} Was correct.
Black shouldn't have feared the potential sacrifice of the knight }}}
\mainline{}
\scalebox{0.90}{\chessboard}\\
\mainline{17. d5 }
\xskakcomment{\small\texttt\justifying{\textcolor{darkgray}{[pgndiagram] A thematic clearance move.Now White is winning. The
way Steinitz concludes the game is highly spectacular}}}
\mainline{17... cxd5 18. Nd4 }
\xskakcomment{\small\texttt\justifying{\textcolor{darkgray}{The
knight is coming to e6 with great effect}}}
\mainline{18... Kf7 }
\scalebox{0.90}{\chessboard}\\
\mainline{19. Ne6 Rhc8 }
\xskakcomment{\small\texttt\justifying{\textcolor{gray}{Variant : \variation{19... Nc6} Was no
better, due to the weakness of d5 pawn }}}
\mainline{20. Qg4 }
\xskakcomment{\small\texttt\justifying{\textcolor{darkgray}{Now the g7 also is weak}}}
\mainline{20... g6 }
\scalebox{0.90}{\chessboard}\\
\mainline{21. Ng5+ }
\xskakcomment{\small\texttt\justifying{\textcolor{darkgray}{[pgndiagram] A highly
attractive finish and one of the most remarkable combinations in the history
of chess.}}}
\mainline{21... Ke8 22. Rxe7+ }
\xskakcomment{\small\texttt\justifying{\textcolor{darkgray}{A fantastic blow. At first sight it seems that
White will overextend himself, because his back rank is weak. But Steinitz has
calculted everything precisely}}}
\mainline{22... Kf8 }
\xskakcomment{\small\texttt\justifying{\textcolor{darkgray}{[pgndiagram] The best chance}}}

\xskakcomment{\small\texttt\justifying{\textcolor{gray}{Variant : \variation{22... Kxe7} }}}
\mainline{}
\scalebox{0.90}{\chessboard}\\
\mainline{23. Rf7+ }
\xskakcomment{\small\texttt\justifying{\textcolor{darkgray}{The combination continues}}}
\mainline{23... Kg8 24. Rg7+ }
\xskakcomment{\small\texttt\justifying{\textcolor{darkgray}{Naturally, now Black can't
take the rook on g7 due to capture on d7 with check}}}
\mainline{24... Kh8 }
\scalebox{0.90}{\chessboard}\\
\mainline{25. Rxh7+ }
\xskakcomment{\small\texttt\justifying{\textcolor{darkgray}{
[pgndiagram] Here Bardeleben left the game. Steinitz willingly demonstrated
the spectators what awaited his opponent had he continued the game}}}
\mainline{25... Kg8 26. Rg7+ Kh8 }
\xskakcomment{\small\texttt\justifying{\textcolor{gray}{Variant : \variation{26... Kf8} }}}
\mainline{}
\scalebox{0.90}{\chessboard}\\
\mainline{27. Qh4+ Kxg7 28. Qh7+ Kf8 }
\scalebox{0.90}{\chessboard}\\
\mainline{29. Qh8+ Ke7 30. Qg7+ Ke8 }
\scalebox{0.90}{\chessboard}\\
\mainline{31. Qg8+ Ke7 32. Qf7+ Kd8 }
\scalebox{0.90}{\chessboard}\\
\mainline{33. Qf8+ Qe8 34. Nf7+ Kd7 }
\scalebox{0.90}{\chessboard}\\
\mainline{35. Qd6# }
\xskakcomment{\small\texttt\justifying{\textcolor{darkgray}{[pgndiagram]
And at the end of the story, everything concludes with an epaulette mate.
Fantastic}}}
\mainline{}
\end{multicols}
\newpage

\chessevent{Steinitz - Chigorin World Championship }
\chessopening{Havana CUB}

Date : 1892.01.07

Round : 4

Result : 1-0

\whitename{Wilhelm Steinitz}

\blackname{Mikhail Chigorin}

\ECO{C65}

\whiteelo{?}

\blackelo{?}

Plycount : 57

\makegametitle
\begin{multicols}{2}
\noindent
\newchessgame[id=main]
\xskakset{style=styleC}
\mainline{1. e4 e5 2. Nf3 Nc6 }
\scalebox{0.90}{\chessboard}\\
\mainline{3. Bb5 Nf6 4. d3 }
\xskakcomment{\small\texttt\justifying{\textcolor{darkgray}{[pgndiagram] The way of avoiding Berlin
even in the 19th century.}}}
\mainline{4... d6 }
\scalebox{0.90}{\chessboard}\\
\mainline{5. c3 g6 }
\xskakcomment{\small\texttt\justifying{\textcolor{darkgray}{The fianchetto against d3 and c3 is
encountered even today on an occasion.}}}
\mainline{6. Nbd2 Bg7 }
\scalebox{0.90}{\chessboard}\\
\mainline{7. Nf1 }
\xskakcomment{\small\texttt\justifying{\textcolor{darkgray}{But I am not a big
fan of this move. White violates one of the basic principles of not moving the
same piece twice in the opening.  Although the position is somewhat slow and
closed enough, I think that he loses some of his flexibility; the knight might
take the alternative Nc3-Ne3 route in the future if neccessary.}}}

\xskakcomment{\small\texttt\justifying{\textcolor{gray}{Variant : \variation{7. O-O} 
Seems more natural }}}
\mainline{7... O-O 8. Ba4 }
\xskakcomment{\small\texttt\justifying{\textcolor{darkgray}{But this loses another tempo}}}

\xskakcomment{\small\texttt\justifying{\textcolor{gray}{Variant : \variation{8. Ng3} Was more consistent }}}
\mainline{8... Nd7 }
\xskakcomment{\small\texttt\justifying{\textcolor{darkgray}{[pgndiagram] Chigorin also moves his knight via c5
to e6, but this removes the knight from the kingside and loses time.}}}

\xskakcomment{\small\texttt\justifying{\textcolor{gray}{Variant : \variation{8... d5} Was possible, exploiting White's delay in the opening. }}}
\mainline{}
\scalebox{0.90}{\chessboard}\\
\mainline{9. Ne3 }
\xskakcomment{\small\texttt\justifying{\textcolor{gray}{Variant : \variation{9. Ng3} Was probably better, in order to
discourage the f5 idea. The point is that on g3 knight doesn't block the
bishop; the g5 square is defended }}}
\mainline{9... Nc5 }
\xskakcomment{\small\texttt\justifying{\textcolor{darkgray}{Chasing the bishop where
it wants to go}}}

\xskakcomment{\small\texttt\justifying{\textcolor{gray}{Variant : \variation{9... f5} Was now interesting }}}
\mainline{10. Bc2 Ne6 }
\scalebox{0.90}{\chessboard}\\
\mainline{11. h4 }
\xskakcomment{\small\texttt\justifying{\textcolor{darkgray}{
[pgndiagram] Great understanding of the game by Steinitz. He recognizes that
Black is not in control in the centre, so he begins a wing attack}}}
\mainline{11... Ne7 }
\xskakcomment{\small\texttt\justifying{\textcolor{darkgray}{And
immediately there comes a mistake. Chigorin plays according to the principles
of the old school: attack on a flank should be countered by the centar blow.
However, White's centre is very stable here}}}

\xskakcomment{\small\texttt\justifying{\textcolor{gray}{Variant : \variation{11... h5} Was the way to go,
although they simply didn't move pawns in front of their king in those times. }}}
\mainline{12. h5 d5 }
\scalebox{0.90}{\chessboard}\\
\mainline{13. hxg6 fxg6 }
\xskakcomment{\small\texttt\justifying{\textcolor{gray}{Variant : \variation{13... hxg6} Leads by force to a dominating endgame
for White }}}
\mainline{14. exd5 Nxd5 }
\scalebox{0.90}{\chessboard}\\
\mainline{15. Nxd5 Qxd5 16. Bb3 }
\xskakcomment{\small\texttt\justifying{\textcolor{darkgray}{[pgndiagram] The b3-g8 diagonal is weak and Steinitz
immediately takes advantage of that}}}
\mainline{16... Qc6 }
\scalebox{0.90}{\chessboard}\\
\mainline{17. Qe2 Bd7 18. Be3 Kh8 }
\scalebox{0.90}{\chessboard}\\
\mainline{19. O-O-O Rae8 20. Qf1 }
\xskakcomment{\small\texttt\justifying{\textcolor{darkgray}{Removing the queen from the bishop battery and preparing the d4
advance. This move is fantastic even if it objectively isn't the best}}}
\mainline{20... a5 }
\xskakcomment{\small\texttt\justifying{\textcolor{darkgray}{
The final mistake.}}}

\xskakcomment{\small\texttt\justifying{\textcolor{gray}{Variant : \variation{20... h5} Black should have taken measures on the
kingside. This move counters White's prepared combination and allows Black to
successfuly defend }}}
\mainline{}
\scalebox{0.90}{\chessboard}\\
\mainline{21. d4 }
\xskakcomment{\small\texttt\justifying{\textcolor{darkgray}{Now Steinitz carries out another remarkable
combination}}}
\mainline{21... exd4 22. Nxd4 Bxd4 }
\xskakcomment{\small\texttt\justifying{\textcolor{darkgray}{[pgndiagram] The spectacular finish is near.
It is a great exercise to stop here and try to find the winning combination on
your own}}}
\mainline{}
\scalebox{0.90}{\chessboard}\\
\mainline{23. Rxd4 Nxd4 24. Rxh7+ }
\xskakcomment{\small\texttt\justifying{\textcolor{darkgray}{[pgndiagram] Black gets mated}}}
\mainline{24... Kxh7 }
\scalebox{0.90}{\chessboard}\\
\mainline{25. Qh1+ Kg7 26. Bh6+ Kf6 }
\scalebox{0.90}{\chessboard}\\
\mainline{27. Qh4+ Ke5 28. Qxd4+ }
\xskakcomment{\small\texttt\justifying{\textcolor{darkgray}{And here Chigorin resigned,
one move before checkmate}}}
\mainline{28... Kf5 }
\scalebox{0.90}{\chessboard}\\
\mainline{29. g4# }
\xskakcomment{\small\texttt\justifying{\textcolor{darkgray}{[pgndiagram] A celebrated game that
demonstrated that in strategical positions Chigorin couldn't compete with
Steinitz on equal terms.}}}
\mainline{}
\end{multicols}
\newpage

\chessevent{London}
\chessopening{London}

Date : 1862.??.??

Round : ?

Result : 1-0

\whitename{Wilhelm Steinitz}

\blackname{Augustus Mongredien}

\ECO{B01}

\whiteelo{?}

\blackelo{?}

Plycount : 57

\makegametitle
\begin{multicols}{2}
\noindent
\newchessgame[id=main]
\xskakset{style=styleC}
\mainline{1. e4 d5 2. exd5 Qxd5 }
\scalebox{0.90}{\chessboard}\\
\mainline{3. Nc3 Qd8 }
\xskakcomment{\small\texttt\justifying{\textcolor{darkgray}{[pgndiagram] It is interesting to see
this line of Scandinavian in a game from 1862.}}}
\mainline{4. d4 e6 }
\xskakcomment{\small\texttt\justifying{\textcolor{darkgray}{I am not an expert
on this variation, but I believe that Black should consider playing some other
moves and wait for an opportunity to pin the knight on f3. Here his light
squared bishop will remain a problem piece}}}

\xskakcomment{\small\texttt\justifying{\textcolor{gray}{Variant : \variation{4... Nf6} }}}
\mainline{}
\scalebox{0.90}{\chessboard}\\
\mainline{5. Nf3 Nf6 6. Bd3 Be7 }
\xskakcomment{\small\texttt\justifying{\textcolor{gray}{Variant : \variation{6... c5} Transposes to the position from the French defense where Black is a tempo
down }}}
\mainline{}
\scalebox{0.90}{\chessboard}\\
\mainline{7. O-O O-O 8. Be3 }
\xskakcomment{\small\texttt\justifying{\textcolor{darkgray}{[pgndiagram] I am not certain
about this move. It does develop a piece, but the bishop on e3 is not optimal.}}}

\xskakcomment{\small\texttt\justifying{\textcolor{gray}{Variant : \variation{8. Ne4} Waiting with the bishop, seems to me to be  more in the spirit of the
position. }}}
\mainline{8... b6 }
\scalebox{0.90}{\chessboard}\\
\mainline{9. Ne5 Bb7 10. f4 }
\xskakcomment{\small\texttt\justifying{\textcolor{darkgray}{Steinitz goes all in, but Black
could have defended}}}
\mainline{10... Nbd7 }
\xskakcomment{\small\texttt\justifying{\textcolor{gray}{Variant : \variation{10... Nc6} Attacking d4 and threatening Nb4, was
interesting }}}
\mainline{}
\scalebox{0.90}{\chessboard}\\
\mainline{11. Qe2 Nd5 }
\xskakcomment{\small\texttt\justifying{\textcolor{gray}{Variant : \variation{11... c5} Was probably better
than the game continuation }}}
\mainline{12. Nxd5 exd5 }
\xskakcomment{\small\texttt\justifying{\textcolor{gray}{Variant : \variation{12... Bxd5} }}}
\mainline{}
\scalebox{0.90}{\chessboard}\\
\mainline{13. Rf3 }
\xskakcomment{\small\texttt\justifying{\textcolor{darkgray}{Now White's pieces enjoy excellent prospects}}}
\mainline{13... f5 }
\xskakcomment{\small\texttt\justifying{\textcolor{darkgray}{Not a
move made willingly}}}

\xskakcomment{\small\texttt\justifying{\textcolor{gray}{Variant : \variation{13... Nf6} }}}
\mainline{14. Rh3 }
\xskakcomment{\small\texttt\justifying{\textcolor{gray}{Variant : \variation{14. Rg3} Computer
indicates that this is the stronger continuation of attack. }}}
\mainline{14... g6 }
\xskakcomment{\small\texttt\justifying{\textcolor{darkgray}{Black has to prevent the move Qh5}}}

\xskakcomment{\small\texttt\justifying{\textcolor{gray}{Variant n°1: \variation{14... Nxe5} }}}

\xskakcomment{\small\texttt\justifying{\textcolor{gray}{Variant n°2: \variation{14... c5} }}}
\mainline{}
\scalebox{0.90}{\chessboard}\\
\mainline{15. g4 }
\xskakcomment{\small\texttt\justifying{\textcolor{darkgray}{Going
for it. The computer doesn't like this move, but for a human this is very
attractive}}}
\mainline{15... fxg4 }
\xskakcomment{\small\texttt\justifying{\textcolor{darkgray}{[pgndiagram] And immediately there comes a big mistake,
which loses}}}

\xskakcomment{\small\texttt\justifying{\textcolor{gray}{Variant : \variation{15... Nxe5} }}}
\mainline{16. Rxh7 }
\xskakcomment{\small\texttt\justifying{\textcolor{darkgray}{[pgndiagram] A brilliant conception and another
rook sacrifice on h7}}}
\mainline{16... Nxe5 }
\xskakcomment{\small\texttt\justifying{\textcolor{gray}{Variant : \variation{16... Kxh7} }}}
\mainline{}
\scalebox{0.90}{\chessboard}\\
\mainline{17. fxe5 Kxh7 }
\xskakcomment{\small\texttt\justifying{\textcolor{gray}{Variant : \variation{17... Bg5} Is another way of losing, in technical instead of the tactical way }}}
\mainline{18. Qxg4 }
\xskakcomment{\small\texttt\justifying{\textcolor{darkgray}{[pgndiagram] We have transposed to the
previously examined variation}}}
\mainline{18... Rg8 }
\xskakcomment{\small\texttt\justifying{\textcolor{darkgray}{Losing even faster than 18... Qe8}}}
\mainline{}
\scalebox{0.90}{\chessboard}\\
\mainline{19. Qh5+ Kg7 20. Qh6+ Kf7 }
\scalebox{0.90}{\chessboard}\\
\mainline{21. Qh7+ Ke6 22. Qh3+ Kf7 }
\scalebox{0.90}{\chessboard}\\
\mainline{23. Rf1+ Ke8 24. Qe6 Rg7 }
\scalebox{0.90}{\chessboard}\\
\mainline{25. Bg5 Qd7 26. Bxg6+ Rxg6 }
\scalebox{0.90}{\chessboard}\\
\mainline{27. Qxg6+ Kd8 28. Rf8+ Qe8 }
\scalebox{0.90}{\chessboard}\\
\mainline{29. Qxe8# }
\xskakcomment{\small\texttt\justifying{\textcolor{darkgray}{[pgndiagram] A very nice
tactical bloodbath, although it is apparent that Steinitz's play was less
strategicaly founded (after all it was year 1862). Compared to the previous
two games, he went all in in this one, but the tactical finish was allowed
only by his opponent's mistakes; they didn't result from his strategical
superiority.}}}
\mainline{}
\end{multicols}
\newpage

\chessevent{London (England)}
\chessopening{London (England)}

Date : 1863.??.??

Round : ?

Result : 1-0

\whitename{Wilhelm Steinitz}

\blackname{Augustus Mongredien}

\ECO{B06}

\whiteelo{?}

\blackelo{?}

Plycount : 43

\makegametitle
\begin{multicols}{2}
\noindent
\newchessgame[id=main]
\xskakset{style=styleC}
\mainline{1. e4 g6 }
\xskakcomment{\small\texttt\justifying{\textcolor{darkgray}{I have chosen this game only because it featured the modern defence}}}
\mainline{2. d4 Bg7 }
\scalebox{0.90}{\chessboard}\\
\mainline{3. c3 }
\xskakcomment{\small\texttt\justifying{\textcolor{darkgray}{This move is death for every Modern player}}}
\mainline{3... b6 }
\xskakcomment{\small\texttt\justifying{\textcolor{darkgray}{And with
this we have entered the uncharted waters.}}}
\mainline{4. Be3 }
\xskakcomment{\small\texttt\justifying{\textcolor{darkgray}{I think that
combination of Be3 and c3 is not good. The d4 square is safe enough. On e3
bishop might become a subject of the Ng4 in the future}}}
\mainline{4... Bb7 }
\scalebox{0.90}{\chessboard}\\
\mainline{5. Nd2 d6 6. Ngf3 e5 }
\xskakcomment{\small\texttt\justifying{\textcolor{darkgray}{[pgndiagram] Generally speaking, the combination of the queenside
fianchetto together with the e5 advance should be condemned.}}}

\xskakcomment{\small\texttt\justifying{\textcolor{gray}{Variant : \variation{6... Nd7} }}}
\mainline{}
\scalebox{0.90}{\chessboard}\\
\mainline{7. dxe5 dxe5 8. Bc4 }
\xskakcomment{\small\texttt\justifying{\textcolor{darkgray}{Now Black can't play Nf6 and has some tactical problems as well}}}
\mainline{8... Ne7 }
\xskakcomment{\small\texttt\justifying{\textcolor{darkgray}{The losing move}}}

\xskakcomment{\small\texttt\justifying{\textcolor{gray}{Variant : \variation{8... Nd7} Was the best, trying to develop }}}
\mainline{}
\scalebox{0.90}{\chessboard}\\
\mainline{9. Qe2 }
\xskakcomment{\small\texttt\justifying{\textcolor{gray}{Variant : \variation{9. Bxf7+} Was winning immediately }}}
\mainline{9... O-O }
\xskakcomment{\small\texttt\justifying{\textcolor{darkgray}{another mistake. Castling
without the knight on f6 and with the rook on h1 is very dangerous}}}

\xskakcomment{\small\texttt\justifying{\textcolor{gray}{Variant : \variation{9... Nc8} 
Is computer's idea, trying to place the knight on d6 }}}
\mainline{10. h4 }
\xskakcomment{\small\texttt\justifying{\textcolor{darkgray}{[pgndiagram] A
familiar advance of Harry the h-pawn}}}
\mainline{10... Nd7 }
\xskakcomment{\small\texttt\justifying{\textcolor{gray}{Variant : \variation{10... h5} Was no good here }}}
\mainline{}
\scalebox{0.90}{\chessboard}\\
\mainline{11. h5 }
\xskakcomment{\small\texttt\justifying{\textcolor{darkgray}{White has a
decisive attack}}}
\mainline{11... Nf6 12. hxg6 }
\xskakcomment{\small\texttt\justifying{\textcolor{gray}{Variant : \variation{12. Nxe5} Was stronger }}}
\mainline{12... Nxg6 }
\xskakcomment{\small\texttt\justifying{\textcolor{darkgray}{Defending the e-pawn}}}

\xskakcomment{\small\texttt\justifying{\textcolor{gray}{Variant : \variation{12... hxg6} }}}
\mainline{}
\scalebox{0.90}{\chessboard}\\
\mainline{13. O-O-O c5 14. Ng5 a6 }
\xskakcomment{\small\texttt\justifying{\textcolor{darkgray}{A final mistake of the game}}}
\mainline{}
\scalebox{0.90}{\chessboard}\\
\mainline{15. Nxh7 }
\xskakcomment{\small\texttt\justifying{\textcolor{darkgray}{[pgndiagram] Another sacrifice on h7 and another correct one}}}
\mainline{15... Nxh7 16. Rxh7 }
\xskakcomment{\small\texttt\justifying{\textcolor{gray}{Variant : \variation{16. Qh5} }}}
\mainline{16... Kxh7 }
\scalebox{0.90}{\chessboard}\\
\mainline{17. Qh5+ Kg8 18. Rh1 Re8 }
\scalebox{0.90}{\chessboard}\\
\mainline{19. Qxg6 Qf6 20. Bxf7+ Qxf7 }
\xskakcomment{\small\texttt\justifying{\textcolor{gray}{Variant : \variation{20... Kf8} [pgndiagram] Also doesn't help }}}
\mainline{}
\scalebox{0.90}{\chessboard}\\
\mainline{21. Rh8+ }
\xskakcomment{\small\texttt\justifying{\textcolor{darkgray}{[pgndiagram]A finishing touch. White wins
the queen and the game}}}
\mainline{21... Kxh8 22. Qxf7 }
\end{multicols}
\newpage

\chessevent{Anderssen - Steinitz}
\chessopening{London ENG}

Date : 1866.08.08

Round : 13

Result : 0-1

\whitename{Adolf Anderssen}

\blackname{Wilhelm Steinitz}

\ECO{C65}

\whiteelo{?}

\blackelo{?}

Plycount : 86

\makegametitle
\begin{multicols}{2}
\noindent
\newchessgame[id=main]
\xskakset{style=styleC}
\mainline{1. e4 e5 2. Nf3 Nc6 }
\scalebox{0.90}{\chessboard}\\
\mainline{3. Bb5 Nf6 4. d3 d6 }
\scalebox{0.90}{\chessboard}\\
\mainline{5. Bxc6+ bxc6 6. h3 g6 }
\xskakcomment{\small\texttt\justifying{\textcolor{darkgray}{[pgndiagram]
These Anti-Berlin lines with kingside fianchetto were championed successfully
by Steinitz}}}
\mainline{}
\scalebox{0.90}{\chessboard}\\
\mainline{7. Nc3 Bg7 8. O-O O-O }
\scalebox{0.90}{\chessboard}\\
\mainline{9. Bg5 h6 10. Be3 c5 }
\scalebox{0.90}{\chessboard}\\
\mainline{11. Rb1 Ne8 12. b4 }
\xskakcomment{\small\texttt\justifying{\textcolor{darkgray}{
[pgndiagram] A very modern interpretation of the opening for those years.
White recognizes that he has to play on the queenside and decides to undouble
the adversary pawns in order to open a file for his rook}}}
\mainline{12... cxb4 }
\scalebox{0.90}{\chessboard}\\
\mainline{13. Rxb4 c5 14. Ra4 }
\xskakcomment{\small\texttt\justifying{\textcolor{darkgray}{Only this is slightly misguided. White removes his rook from the open
file}}}

\xskakcomment{\small\texttt\justifying{\textcolor{gray}{Variant : \variation{14. Rb1} }}}
\mainline{14... Bd7 }
\scalebox{0.90}{\chessboard}\\
\mainline{15. Ra3 f5 16. Qb1 Kh8 }
\xskakcomment{\small\texttt\justifying{\textcolor{gray}{Variant : \variation{16... Nc7} Preventing Qb7, came into consideration }}}
\mainline{}
\scalebox{0.90}{\chessboard}\\
\mainline{17. Qb7 a5 18. Rb1 a4 }
\scalebox{0.90}{\chessboard}\\
\mainline{19. Qd5 }
\xskakcomment{\small\texttt\justifying{\textcolor{darkgray}{[pgndiagram] From this moment onwards,
White's game starts going downhill rapidly. His pieces are stuck on the
queenside and are unable to join the kingside and defend the king}}}

\xskakcomment{\small\texttt\justifying{\textcolor{gray}{Variant : \variation{19. Qb6} 
Going for the queen exchange, was better. Black can't really avoid it }}}
\mainline{19... Qc8 20. Rb6 Ra7 }
\scalebox{0.90}{\chessboard}\\
\mainline{21. Kh2 f4 }
\xskakcomment{\small\texttt\justifying{\textcolor{darkgray}{[pgndiagram] Black finally decides
on this move. He will soon set his pawn chain in motion, and alraedy it is
very hard to suggest how White should counter that plan}}}
\mainline{22. Bd2 g5 }
\scalebox{0.90}{\chessboard}\\
\mainline{23. Qc4 Qd8 }
\xskakcomment{\small\texttt\justifying{\textcolor{darkgray}{Defending g5 and intending h5}}}
\mainline{24. Rb1 }
\xskakcomment{\small\texttt\justifying{\textcolor{darkgray}{Going from frying pan to fire}}}

\xskakcomment{\small\texttt\justifying{\textcolor{gray}{Variant : \variation{24. Nd5} White had to seek tactical countermeasures. The point of this knight jump
is to open the diagonal for the slumbering bishop on d2 }}}
\mainline{24... Nf6 }
\xskakcomment{\small\texttt\justifying{\textcolor{darkgray}{But anyway, White's position is unpleasant}}}
\mainline{}
\scalebox{0.90}{\chessboard}\\
\mainline{25. Kg1 }
\xskakcomment{\small\texttt\justifying{\textcolor{gray}{Variant : \variation{25. Nxa4} }}}
\mainline{25... Nh7 }
\xskakcomment{\small\texttt\justifying{\textcolor{darkgray}{h5-g4 is coming soon}}}
\mainline{26. Kf1 }
\xskakcomment{\small\texttt\justifying{\textcolor{darkgray}{White can't
escape with the king}}}

\xskakcomment{\small\texttt\justifying{\textcolor{gray}{Variant : \variation{26. Qd5} Was the last chance }}}
\mainline{26... h5 }
\scalebox{0.90}{\chessboard}\\
\mainline{27. Ng1 g4 }
\xskakcomment{\small\texttt\justifying{\textcolor{darkgray}{[pgndiagram] Now Black's attack is
simply irresistible}}}
\mainline{28. hxg4 }
\xskakcomment{\small\texttt\justifying{\textcolor{gray}{Variant : \variation{28. f3} Is also hopeless }}}
\mainline{28... hxg4 }
\scalebox{0.90}{\chessboard}\\
\mainline{29. f3 Qh4 30. Nd1 Ng5 }
\xskakcomment{\small\texttt\justifying{\textcolor{darkgray}{[pgndiagram] All Black's pieces join the
attack and White doesn't have enough space to bring in his reserves.}}}
\mainline{}
\scalebox{0.90}{\chessboard}\\
\mainline{31. Be1 Qh2 32. d4 }
\xskakcomment{\small\texttt\justifying{\textcolor{darkgray}{This merely hastens the end, but White was doomed anyway. The rest
doesn't require any commentary}}}
\mainline{32... gxf3 }
\scalebox{0.90}{\chessboard}\\
\mainline{33. gxf3 Nh3 34. Bf2 Nxg1 }
\scalebox{0.90}{\chessboard}\\
\mainline{35. dxc5 Qh3+ 36. Ke1 Nxf3+ }
\scalebox{0.90}{\chessboard}\\
\mainline{37. Rxf3 Qxf3 38. Nc3 dxc5 }
\scalebox{0.90}{\chessboard}\\
\mainline{39. Bxc5 Rc7 40. Nd5 Rxc5 }
\scalebox{0.90}{\chessboard}\\
\mainline{41. Qxc5 Qxe4+ 42. Kf2 Rc8 }
\scalebox{0.90}{\chessboard}\\
\mainline{43. Nc7 Qe3+ }
\xskakcomment{\small\texttt\justifying{\textcolor{darkgray}{[pgndiagram] A fantastic win, demonstrating
Steinitz's superiority over his contemporaries in the strategical positions}}}
\mainline{}
\end{multicols}
\newpage

\chessevent{Steinitz - Blackburne}
\chessopening{London ENG}

Date : 1876.02.29

Round : 6

Result : 0-1

\whitename{Joseph Henry Blackburne}

\blackname{Wilhelm Steinitz}

\ECO{C45}

\whiteelo{?}

\blackelo{?}

Plycount : 132

\makegametitle
\begin{multicols}{2}
\noindent
\newchessgame[id=main]
\xskakset{style=styleC}
\mainline{1. e4 e5 2. Nf3 Nc6 }
\scalebox{0.90}{\chessboard}\\
\mainline{3. d4 exd4 4. Nxd4 Qh4 }
\xskakcomment{\small\texttt\justifying{\textcolor{darkgray}{[pgndiagram] This variation of the
Scotch defence is nowadays considered dangerous. But Steinitz simply grabs the
pawn and defends it for the remainder of the game.}}}
\mainline{}
\scalebox{0.90}{\chessboard}\\
\mainline{5. Nb5 }
\xskakcomment{\small\texttt\justifying{\textcolor{gray}{Variant : \variation{5. Nc3} Is the
modern move }}}
\mainline{5... Bb4+ 6. Bd2 Qxe4+ }
\scalebox{0.90}{\chessboard}\\
\mainline{7. Be2 Kd8 }
\xskakcomment{\small\texttt\justifying{\textcolor{darkgray}{[pgndiagram] Black has given up a pawn for the right to
castle. Objectively speaking, White should have decent compensation for the
pawn, but it is not so easy to prove it.}}}
\mainline{8. O-O Bxd2 }
\scalebox{0.90}{\chessboard}\\
\mainline{9. Qxd2 }
\xskakcomment{\small\texttt\justifying{\textcolor{gray}{Variant : \variation{9. Nxd2} 
Already here, what could be more natural than bringing another piece in the
game? }}}
\mainline{9... a6 }
\xskakcomment{\small\texttt\justifying{\textcolor{darkgray}{Now Black has time to chase the knight away from d4}}}
\mainline{10. N5a3 }
\xskakcomment{\small\texttt\justifying{\textcolor{gray}{Variant : \variation{10. N1c3} Was better, not allowing Qd4 }}}
\mainline{10... Qd4 }
\scalebox{0.90}{\chessboard}\\
\mainline{11. Qg5+ Qf6 12. Qd2 }
\xskakcomment{\small\texttt\justifying{\textcolor{darkgray}{[pgndiagram] Giving another pawn away is
too much}}}

\xskakcomment{\small\texttt\justifying{\textcolor{gray}{Variant : \variation{12. Qc1} Is ugly, but neccessary. It is clear that White has lost
the thread and that Black is taking over, but the game continues }}}
\mainline{12... Qxb2 }
\scalebox{0.90}{\chessboard}\\
\mainline{13. Nc4 Qd4 }
\xskakcomment{\small\texttt\justifying{\textcolor{gray}{Variant : \variation{13... Qxa1} }}}
\mainline{14. Qc1 Nge7 }
\xskakcomment{\small\texttt\justifying{\textcolor{gray}{Variant : \variation{14... Qxa1} Was now possible }}}
\mainline{}
\scalebox{0.90}{\chessboard}\\
\mainline{15. Nbd2 d6 }
\xskakcomment{\small\texttt\justifying{\textcolor{darkgray}{
[pgndiagram] Even so, Black has retained the advantage. Two pawns are too much}}}
\mainline{16. Rd1 Be6 }
\scalebox{0.90}{\chessboard}\\
\mainline{17. Qa3 Nd5 18. Nb3 Qc3 }
\scalebox{0.90}{\chessboard}\\
\mainline{19. Bf1 }
\xskakcomment{\small\texttt\justifying{\textcolor{gray}{Variant : \variation{19. Nxd6} Was the last chance }}}
\mainline{19... Ndb4 }
\xskakcomment{\small\texttt\justifying{\textcolor{darkgray}{
[pgndiagram]Black has consolidated completely and now only needs to activate
his rooks}}}
\mainline{20. Ne3 Re8 }
\scalebox{0.90}{\chessboard}\\
\mainline{21. Rd2 }
\xskakcomment{\small\texttt\justifying{\textcolor{darkgray}{Desperation. White embarks on a faulty
combination}}}
\mainline{21... Bxb3 22. Rad1 Rxe3 }
\xskakcomment{\small\texttt\justifying{\textcolor{darkgray}{[pgndiagram] A nice tactical refutation}}}
\mainline{}
\scalebox{0.90}{\chessboard}\\
\mainline{23. fxe3 Nxc2 24. Qc1 }
\xskakcomment{\small\texttt\justifying{\textcolor{gray}{Variant : \variation{24. Qxb3} Was the best, transposing into the lost
endgame }}}
\mainline{24... Qxe3+ }
\scalebox{0.90}{\chessboard}\\
\mainline{25. Kh1 Ba4 }
\xskakcomment{\small\texttt\justifying{\textcolor{darkgray}{[pgndiagram] Now Black is
easily winning (a piece and four pawns for the rook).}}}
\mainline{26. Bc4 N2d4 }
\scalebox{0.90}{\chessboard}\\
\mainline{27. Re1 Qf4 28. Rf1 Qh6 }
\scalebox{0.90}{\chessboard}\\
\mainline{29. Qb2 Qe3 30. Bxf7 Bb5 }
\scalebox{0.90}{\chessboard}\\
\mainline{31. Rfd1 Nf5 32. a4 Ne5 }
\scalebox{0.90}{\chessboard}\\
\mainline{33. axb5 Nxf7 34. Re2 Qh6 }
\scalebox{0.90}{\chessboard}\\
\mainline{35. Qb3 axb5 36. g4 Nd4 }
\scalebox{0.90}{\chessboard}\\
\mainline{37. Rxd4 Ra1+ 38. Kg2 Qf6 }
\scalebox{0.90}{\chessboard}\\
\mainline{39. Rde4 Ne5 40. Rf2 Qg6 }
\scalebox{0.90}{\chessboard}\\
\mainline{41. Ref4 c6 42. Qe3 Kc7 }
\scalebox{0.90}{\chessboard}\\
\mainline{43. h3 h5 44. Rf5 hxg4 }
\scalebox{0.90}{\chessboard}\\
\mainline{45. Rg5 }
\xskakcomment{\small\texttt\justifying{\textcolor{darkgray}{[pgndiagram]
Steinitz has complicated his task somewhat, but here he is clearly winning}}}
\mainline{45... gxh3+ }
\xskakcomment{\small\texttt\justifying{\textcolor{darkgray}{But suddenly, this is a huge mistake, after which Black is probably
not even better}}}

\xskakcomment{\small\texttt\justifying{\textcolor{gray}{Variant : \variation{45... Qh6} Was easily winning }}}
\mainline{46. Kh2 Ra3 }
\xskakcomment{\small\texttt\justifying{\textcolor{gray}{Variant : \variation{46... Qe8} Was comparatively the best }}}
\mainline{}
\scalebox{0.90}{\chessboard}\\
\mainline{47. Qxe5 dxe5 48. Rxg6 }
\xskakcomment{\small\texttt\justifying{\textcolor{darkgray}{
[pgndiagram] Now White is a full rook up and at least shouldn't lose}}}
\mainline{48... b4 }
\scalebox{0.90}{\chessboard}\\
\mainline{49. Rb2 }
\xskakcomment{\small\texttt\justifying{\textcolor{gray}{Variant : \variation{49. Rf7+} }}}
\mainline{49... c5 50. Rf2 Rd3 }
\scalebox{0.90}{\chessboard}\\
\mainline{51. Rc2 }
\xskakcomment{\small\texttt\justifying{\textcolor{darkgray}{A step in the wrong direction, enticing Black to make a move he would like to
do anyway. Also, White pasivizes the rook}}}

\xskakcomment{\small\texttt\justifying{\textcolor{gray}{Variant : \variation{51. Re2} Forcing Black to defend
the pawn }}}
\mainline{51... b6 52. Re6 }
\xskakcomment{\small\texttt\justifying{\textcolor{gray}{Variant : \variation{52. Rf2} Was the
last chance }}}
\mainline{52... b3 }
\xskakcomment{\small\texttt\justifying{\textcolor{darkgray}{With this tempo, Black's pawns are simply
too strong}}}
\mainline{}
\scalebox{0.90}{\chessboard}\\
\mainline{53. Rb2 c4 54. Rxe5 Kc6 }
\scalebox{0.90}{\chessboard}\\
\mainline{55. Rg5 Rd5 56. Rxg7 Kc5 }
\scalebox{0.90}{\chessboard}\\
\mainline{57. Kxh3 Kb4 58. Rb1 b5 }
\scalebox{0.90}{\chessboard}\\
\mainline{59. Rg4 Rd2 60. Rg5 b2 }
\scalebox{0.90}{\chessboard}\\
\mainline{61. Kg3 c3 62. Kf3 Kc4 }
\scalebox{0.90}{\chessboard}\\
\mainline{63. Rgg1 Kb3 64. Ke3 Rd8 }
\scalebox{0.90}{\chessboard}\\
\mainline{65. Rbf1 c2 66. Ke2 Ka2 }
\xskakcomment{\small\texttt\justifying{\textcolor{darkgray}{[pgndiagram] Despite the error in conversion, this
was a masterful defensive effort by Steinitz.}}}
\mainline{}
\end{multicols}
\newpage

\chessevent{Vienna}
\chessopening{Vienna AUT}

Date : 1873.08.12

Round : 7.2

Result : 0-1

\whitename{Adolf Anderssen}

\blackname{Wilhelm Steinitz}

\ECO{C77}

\whiteelo{?}

\blackelo{?}

Plycount : 90

\makegametitle
\begin{multicols}{2}
\noindent
\newchessgame[id=main]
\xskakset{style=styleC}
\mainline{1. e4 e5 2. Nf3 Nc6 }
\scalebox{0.90}{\chessboard}\\
\mainline{3. Bb5 a6 }
\xskakcomment{\small\texttt\justifying{\textcolor{darkgray}{One of the rare instances of Steinitz starting
with 3.. a6 instead of 3... Nf6 or 3... d6}}}
\mainline{4. Ba4 Nf6 }
\scalebox{0.90}{\chessboard}\\
\mainline{5. d3 }
\xskakcomment{\small\texttt\justifying{\textcolor{darkgray}{But still,
Anderssen chooses this move, leading to an Anti-Berlin, like position.}}}
\mainline{5... d6 6. Bxc6+ bxc6 }
\scalebox{0.90}{\chessboard}\\
\mainline{7. h3 g6 }
\xskakcomment{\small\texttt\justifying{\textcolor{darkgray}{[pgndiagram] And again we have the fianchetto and the
exhcange on c6, similarly to the previously examined encounter}}}
\mainline{8. Nc3 Bg7 }
\scalebox{0.90}{\chessboard}\\
\mainline{9. Be3 Rb8 }
\xskakcomment{\small\texttt\justifying{\textcolor{darkgray}{This time Steinitz doesn't go for c5 but seizes the b-file himself}}}
\mainline{10. b3 c5 }
\scalebox{0.90}{\chessboard}\\
\mainline{11. Qd2 h6 }
\xskakcomment{\small\texttt\justifying{\textcolor{darkgray}{It is doubtful whether this move is necessary.}}}

\xskakcomment{\small\texttt\justifying{\textcolor{gray}{Variant : \variation{11... O-O} And it is doubtful whether White wants to exchange the dark squared
bishops; he can't follow it up with attack }}}
\mainline{12. g4 }
\xskakcomment{\small\texttt\justifying{\textcolor{darkgray}{[pgndiagram] But this is completely uncalled for}}}

\xskakcomment{\small\texttt\justifying{\textcolor{gray}{Variant : \variation{12. O-O} Was more natural,
leading to a balanced game }}}
\mainline{12... Ng8 }
\xskakcomment{\small\texttt\justifying{\textcolor{darkgray}{Very strong play by Steinitz. He
starts moving his knight toward d4 immediately}}}

\xskakcomment{\small\texttt\justifying{\textcolor{gray}{Variant : \variation{12... h5} Also came into
consideration }}}
\mainline{}
\scalebox{0.90}{\chessboard}\\
\mainline{13. O-O-O Ne7 14. Ne2 }
\xskakcomment{\small\texttt\justifying{\textcolor{darkgray}{This knight is going nowhere.
Although the position is mildly unpleasant for White, notice how quickly
Steinitz will outplay his opponent}}}

\xskakcomment{\small\texttt\justifying{\textcolor{gray}{Variant : \variation{14. Rdg1} }}}
\mainline{14... Nc6 }
\scalebox{0.90}{\chessboard}\\
\mainline{15. Qc3 }
\xskakcomment{\small\texttt\justifying{\textcolor{darkgray}{The
queen is completely missplaced here}}}
\mainline{15... Nd4 16. Nfg1 O-O }
\scalebox{0.90}{\chessboard}\\
\mainline{17. Ng3 Be6 18. N1e2 Qd7 }
\scalebox{0.90}{\chessboard}\\
\mainline{19. Bxd4 cxd4 20. Qb2 a5 }
\xskakcomment{\small\texttt\justifying{\textcolor{darkgray}{[pgndiagram] Black has once again gained a strong
attack, while White's counterintuitive on the other wing is non existent}}}
\mainline{}
\scalebox{0.90}{\chessboard}\\
\mainline{21. Kd2 }
\xskakcomment{\small\texttt\justifying{\textcolor{gray}{Variant : \variation{21. f4} Was nevertheless better }}}
\mainline{21... d5 22. f3 Qe7 }
\scalebox{0.90}{\chessboard}\\
\mainline{23. Rdf1 Qb4+ 24. Kd1 a4 }
\scalebox{0.90}{\chessboard}\\
\mainline{25. Rh2 c5 26. Nc1 c4 }
\scalebox{0.90}{\chessboard}\\
\mainline{27. a3 Qe7 28. b4 c3 }
\xskakcomment{\small\texttt\justifying{\textcolor{darkgray}{[pgndiagram] White's queen is
completely sidelined and Black obtains free hands on the kingside. It is very
instructive to watch how Steinitz uses his greater control of space to quickly
shift his attention to the other side of the board}}}
\mainline{}
\scalebox{0.90}{\chessboard}\\
\mainline{29. Qa1 Qg5 30. Rff2 f5 }
\scalebox{0.90}{\chessboard}\\
\mainline{31. exf5 }
\xskakcomment{\small\texttt\justifying{\textcolor{darkgray}{[pgndiagram] Opening the kingside. White is completely lost}}}
\mainline{31... gxf5 32. h4 Qg6 }
\scalebox{0.90}{\chessboard}\\
\mainline{33. Nxf5 }
\xskakcomment{\small\texttt\justifying{\textcolor{darkgray}{Losing immediately}}}

\xskakcomment{\small\texttt\justifying{\textcolor{gray}{Variant : \variation{33. gxf5} Was slightly more prudent }}}
\mainline{33... Bxf5 34. gxf5 Rxf5 }
\xskakcomment{\small\texttt\justifying{\textcolor{gray}{Variant : \variation{34... Qg1+} }}}
\mainline{}
\scalebox{0.90}{\chessboard}\\
\mainline{35. Ne2 Rbf8 36. Qa2 Qf7 }
\scalebox{0.90}{\chessboard}\\
\mainline{37. Rh3 Kh7 38. Ng1 }
\xskakcomment{\small\texttt\justifying{\textcolor{gray}{Variant : \variation{38. Kc1} Would
prolong the resistance }}}
\mainline{38... Bf6 }
\scalebox{0.90}{\chessboard}\\
\mainline{39. Ke2 Rg8 }
\xskakcomment{\small\texttt\justifying{\textcolor{gray}{Variant : \variation{39... e4} }}}
\mainline{40. Kf1 Be7 }
\scalebox{0.90}{\chessboard}\\
\mainline{41. Ne2 Rh5 }
\xskakcomment{\small\texttt\justifying{\textcolor{gray}{Variant : \variation{41... Bxh4} }}}
\mainline{42. f4 Bxh4 }
\scalebox{0.90}{\chessboard}\\
\mainline{43. Rff3 e4 44. dxe4 Qg6 }
\scalebox{0.90}{\chessboard}\\
\mainline{45. Ng3 Bxg3 }
\xskakcomment{\small\texttt\justifying{\textcolor{darkgray}{
[pgndiagram] An impressive victory, indicating once again the difference in
the level of understanding}}}
\mainline{}
\end{multicols}
\newpage

\chessevent{Steinitz - Zukertort World Championship}
\chessopening{New York, NY USA}

Date : 1886.01.11

Round : 1

Result : 0-1

\whitename{Johannes Zukertort}

\blackname{Wilhelm Steinitz}

\ECO{D11}

\whiteelo{?}

\blackelo{?}

Plycount : 92

\makegametitle
\begin{multicols}{2}
\noindent
\newchessgame[id=main]
\xskakset{style=styleC}
\mainline{1. d4 }
\xskakcomment{\small\texttt\justifying{\textcolor{darkgray}{It was rare to see queen's pawn moving on the first move in those times}}}
\mainline{1... d5 2. c4 c6 }
\xskakcomment{\small\texttt\justifying{\textcolor{darkgray}{The Slav! In 1886!}}}
\mainline{}
\scalebox{0.90}{\chessboard}\\
\mainline{3. e3 Bf5 4. Nc3 e6 }
\scalebox{0.90}{\chessboard}\\
\mainline{5. Nf3 Nd7 6. a3 }
\xskakcomment{\small\texttt\justifying{\textcolor{darkgray}{A
sign of the times. Obviously there is a lot of theory and lot of possibilities.
I am sure I could write a good book on all the nuances and move orders here.}}}
\mainline{6... Bd6 }
\xskakcomment{\small\texttt\justifying{\textcolor{darkgray}{Provoking White's next move}}}
\mainline{}
\scalebox{0.90}{\chessboard}\\
\mainline{7. c5 }
\xskakcomment{\small\texttt\justifying{\textcolor{darkgray}{Tempting but faulty. Black will
be able to break the bind by the timely e5 move}}}

\xskakcomment{\small\texttt\justifying{\textcolor{gray}{Variant : \variation{7. Be2} Is more natural }}}
\mainline{7... Bc7 8. b4 e5 }
\xskakcomment{\small\texttt\justifying{\textcolor{darkgray}{Thematic in similar structures. White has a
bind on the queenside; Black has to break in the centre}}}
\mainline{}
\scalebox{0.90}{\chessboard}\\
\mainline{9. Be2 }
\xskakcomment{\small\texttt\justifying{\textcolor{gray}{Variant : \variation{9. dxe5} 
[pgndiagram] Was better here, not allowing e4 and freeing the d4 square for
the knight }}}
\mainline{9... Ngf6 10. Bb2 e4 }
\scalebox{0.90}{\chessboard}\\
\mainline{11. Nd2 h5 }
\xskakcomment{\small\texttt\justifying{\textcolor{darkgray}{
Steinitz once again plays principled chess. He gains space on the kingside and
intends to mobilize his every single piece on that area of the board}}}
\mainline{12. h3 Nf8 }
\scalebox{0.90}{\chessboard}\\
\mainline{13. a4 Ng6 14. b5 Nh4 }
\scalebox{0.90}{\chessboard}\\
\mainline{15. g3 Ng2+ }
\xskakcomment{\small\texttt\justifying{\textcolor{darkgray}{[pgndiagram] To my mind this is a
fantastic move. Had such a conception been played in the 21st century,
everyone would assume it was computer preparation. However, it was 1886... 
Steinitz embarks on a modern knight sacrifice, for which he gains long term
compensation}}}
\mainline{16. Kf1 Nxe3+ }
\xskakcomment{\small\texttt\justifying{\textcolor{darkgray}{And here we go. Once again we see the concept
of space advantage utilized. White's pieces, stuck on the queenside will be
unable to come and defend White's king, who is stuck on the kingside}}}
\mainline{}
\scalebox{0.90}{\chessboard}\\
\mainline{17. fxe3 Bxg3 18. Kg2 Bc7 }
\xskakcomment{\small\texttt\justifying{\textcolor{darkgray}{[pgndiagram] This is probably faulty retreat, as it gives
White enough time to organize}}}

\xskakcomment{\small\texttt\justifying{\textcolor{gray}{Variant : \variation{18... h4} }}}
\mainline{}
\scalebox{0.90}{\chessboard}\\
\mainline{19. Qg1 }
\xskakcomment{\small\texttt\justifying{\textcolor{darkgray}{Returning the favour}}}

\xskakcomment{\small\texttt\justifying{\textcolor{gray}{Variant : \variation{19. Qf1} Gaining tempo on the bishop for the
vacation of the king was required }}}
\mainline{19... Rh6 20. Kf1 Rg6 }
\scalebox{0.90}{\chessboard}\\
\mainline{21. Qf2 }
\xskakcomment{\small\texttt\justifying{\textcolor{darkgray}{[pgndiagram] Now White
doesn't have enough time. h3 pawn falls as well}}}
\mainline{21... Qd7 22. bxc6 bxc6 }
\scalebox{0.90}{\chessboard}\\
\mainline{23. Rg1 }
\xskakcomment{\small\texttt\justifying{\textcolor{darkgray}{
White has to part with the h3 pawn}}}

\xskakcomment{\small\texttt\justifying{\textcolor{gray}{Variant : \variation{23. h4} }}}
\mainline{23... Bxh3+ 24. Ke1 Ng4 }
\xskakcomment{\small\texttt\justifying{\textcolor{darkgray}{[pgndiagram] And just like that, Black is simply winning. Bear in
mind that Zuketort was the strongest player apart from Steinitz at that moment.
}}}
\mainline{}
\scalebox{0.90}{\chessboard}\\
\mainline{25. Bxg4 Bxg4 26. Ne2 Qe7 }
\scalebox{0.90}{\chessboard}\\
\mainline{27. Nf4 Rh6 28. Bc3 g5 }
\scalebox{0.90}{\chessboard}\\
\mainline{29. Ne2 Rf6 30. Qg2 }
\xskakcomment{\small\texttt\justifying{\textcolor{darkgray}{
[pgndiagram] White is getting steamrolled}}}
\mainline{30... Rf3 }
\scalebox{0.90}{\chessboard}\\
\mainline{31. Nf1 Rb8 32. Kd2 f5 }
\xskakcomment{\small\texttt\justifying{\textcolor{gray}{Variant : \variation{32... Bh3} Was winning immediately }}}
\mainline{}
\scalebox{0.90}{\chessboard}\\
\mainline{33. a5 f4 34. Rh1 Qf7 }
\scalebox{0.90}{\chessboard}\\
\mainline{35. Re1 fxe3+ 36. Nxe3 Rf2 }
\scalebox{0.90}{\chessboard}\\
\mainline{37. Qxf2 Qxf2 38. Nxg4 Bf4+ }
\scalebox{0.90}{\chessboard}\\
\mainline{39. Kc2 hxg4 40. Bd2 e3 }
\scalebox{0.90}{\chessboard}\\
\mainline{41. Bc1 Qg2 42. Kc3 Kd7 }
\scalebox{0.90}{\chessboard}\\
\mainline{43. Rh7+ Ke6 44. Rh6+ Kf5 }
\scalebox{0.90}{\chessboard}\\
\mainline{45. Bxe3 Bxe3 46. Rf1+ Bf4 }
\xskakcomment{\small\texttt\justifying{\textcolor{darkgray}{[pgndiagram] One of the most underestimated games of all time. I have never
seen it before investigating for the article. Steinitz's conception and
position queen sacrifice was much ahead of his time.}}}
\mainline{}
\end{multicols}
\newpage

\chessevent{Steinitz - Zukertort World Championship}
\chessopening{New Orleans, LA USA}

Date : 1886.03.24

Round : 19

Result : 0-1

\whitename{Johannes Zukertort}

\blackname{Wilhelm Steinitz}

\ECO{D53}

\whiteelo{?}

\blackelo{?}

Plycount : 58

\makegametitle
\begin{multicols}{2}
\noindent
\newchessgame[id=main]
\xskakset{style=styleC}
\mainline{1. d4 d5 2. c4 e6 }
\scalebox{0.90}{\chessboard}\\
\mainline{3. Nc3 Nf6 }
\xskakcomment{\small\texttt\justifying{\textcolor{darkgray}{[pgndiagram] Steinitz deviates from the Slav
which had brought him success earlier in the match}}}
\mainline{4. Bg5 Be7 }
\scalebox{0.90}{\chessboard}\\
\mainline{5. Nf3 O-O 6. c5 }
\xskakcomment{\small\texttt\justifying{\textcolor{darkgray}{[pgndiagram] The best proof that queen's pawn openings were terra
incognita for Zukertort. This move gives Black immediate targets.}}}
\mainline{6... b6 }
\scalebox{0.90}{\chessboard}\\
\mainline{7. b4 bxc5 }
\xskakcomment{\small\texttt\justifying{\textcolor{darkgray}{Returning the favour}}}

\xskakcomment{\small\texttt\justifying{\textcolor{gray}{Variant : \variation{7... a5} Was very strong }}}
\mainline{8. dxc5 }
\xskakcomment{\small\texttt\justifying{\textcolor{darkgray}{The errors
continue}}}

\xskakcomment{\small\texttt\justifying{\textcolor{gray}{Variant : \variation{8. bxc5} Would give White a playable position }}}
\mainline{8... a5 }
\scalebox{0.90}{\chessboard}\\
\mainline{9. a3 d4 }
\xskakcomment{\small\texttt\justifying{\textcolor{darkgray}{[pgndiagram] Black's position is so
good, that virtually everything works for him.}}}

\xskakcomment{\small\texttt\justifying{\textcolor{gray}{Variant : \variation{9... Ne4} This thematic move
is once again strong }}}
\mainline{10. Bxf6 gxf6 }
\scalebox{0.90}{\chessboard}\\
\mainline{11. Na4 e5 }
\xskakcomment{\small\texttt\justifying{\textcolor{gray}{Variant : \variation{11... axb4} }}}
\mainline{12. b5 }
\xskakcomment{\small\texttt\justifying{\textcolor{darkgray}{Now White manages to advance his
pawns and is probably not even worse}}}
\mainline{12... Be6 }
\scalebox{0.90}{\chessboard}\\
\mainline{13. g3 }
\xskakcomment{\small\texttt\justifying{\textcolor{darkgray}{Probably it was wiser not
to allow c6}}}

\xskakcomment{\small\texttt\justifying{\textcolor{gray}{Variant : \variation{13. e3} undermining the centre was decent alternative }}}
\mainline{13... c6 14. bxc6 }
\xskakcomment{\small\texttt\justifying{\textcolor{gray}{Variant : \variation{14. b6} Would lose the c5 pawn }}}
\mainline{14... Nxc6 }
\xskakcomment{\small\texttt\justifying{\textcolor{darkgray}{[pgndiagram] But
now once again White's position is highly unpleasant}}}
\mainline{}
\scalebox{0.90}{\chessboard}\\
\mainline{15. Bg2 Rb8 }
\xskakcomment{\small\texttt\justifying{\textcolor{darkgray}{Threatening
Bb3 and not allowing White to castle}}}
\mainline{16. Qc1 }
\xskakcomment{\small\texttt\justifying{\textcolor{gray}{Variant : \variation{16. Nd2} Was comparatively
better }}}
\mainline{16... d3 }
\xskakcomment{\small\texttt\justifying{\textcolor{darkgray}{[pgndiagram] Now White gets murdered}}}
\mainline{}
\scalebox{0.90}{\chessboard}\\
\mainline{17. e3 e4 18. Nd2 f5 }
\scalebox{0.90}{\chessboard}\\
\mainline{19. O-O Re8 20. f3 Nd4 }
\xskakcomment{\small\texttt\justifying{\textcolor{darkgray}{A nice touch.}}}
\mainline{}
\scalebox{0.90}{\chessboard}\\
\mainline{21. exd4 Qxd4+ 22. Kh1 e3 }
\xskakcomment{\small\texttt\justifying{\textcolor{darkgray}{A cold shower}}}

\xskakcomment{\small\texttt\justifying{\textcolor{gray}{Variant : \variation{22... Qxa4} Possibly Zukertort was hoping for immediate piece recapture }}}
\mainline{}
\scalebox{0.90}{\chessboard}\\
\mainline{23. Nc3 Bf6 }
\xskakcomment{\small\texttt\justifying{\textcolor{darkgray}{Another strong intermediate move}}}
\mainline{24. Ndb1 d2 }
\xskakcomment{\small\texttt\justifying{\textcolor{darkgray}{[pgndiagram] The pawn glide forward.}}}
\mainline{}
\scalebox{0.90}{\chessboard}\\
\mainline{25. Qc2 Bb3 26. Qxf5 d1=Q }
\scalebox{0.90}{\chessboard}\\
\mainline{27. Nxd1 Bxd1 28. Nc3 e2 }
\scalebox{0.90}{\chessboard}\\
\mainline{29. Raxd1 Qxc3 }
\xskakcomment{\small\texttt\justifying{\textcolor{darkgray}{[pgndiagram] A very uneven, irrational
encounter in which Zukertort lost virtually in the opening.}}}
\mainline{}
\end{multicols}
\newpage

\chessevent{Steinitz - Zukertort World Championship}
\chessopening{New Orleans, LA USA}

Date : 1886.03.29

Round : 20

Result : 1-0

\whitename{Wilhelm Steinitz}

\blackname{Johannes Zukertort}

\ECO{C25}

\whiteelo{?}

\blackelo{?}

Plycount : 37

\makegametitle
\begin{multicols}{2}
\noindent
\newchessgame[id=main]
\xskakset{style=styleC}
\mainline{1. e4 e5 2. Nc3 Nc6 }
\scalebox{0.90}{\chessboard}\\
\mainline{3. f4 exf4 4. d4 }
\xskakcomment{\small\texttt\justifying{\textcolor{darkgray}{Steinitz has developed this audacious
gambit in his youth. It is interesting to see playing it in the most important
moment of his career - the World Championship match.}}}
\mainline{4... d5 }
\xskakcomment{\small\texttt\justifying{\textcolor{darkgray}{Playing d5 before
going for the check is worse, since White can take on f4 now as well}}}

\xskakcomment{\small\texttt\justifying{\textcolor{gray}{Variant : \variation{4... Qh4+} }}}
\mainline{}
\scalebox{0.90}{\chessboard}\\
\mainline{5. exd5 }
\xskakcomment{\small\texttt\justifying{\textcolor{darkgray}{Allowing the
check, naturally}}}

\xskakcomment{\small\texttt\justifying{\textcolor{gray}{Variant : \variation{5. Bxf4} }}}
\mainline{5... Qh4+ 6. Ke2 Qe7+ }
\scalebox{0.90}{\chessboard}\\
\mainline{7. Kf2 Qh4+ 8. g3 fxg3+ }
\scalebox{0.90}{\chessboard}\\
\mainline{9. Kg2 }
\xskakcomment{\small\texttt\justifying{\textcolor{darkgray}{[pgndiagram] Actually, this way of
refuting the gambit is probably not the most energetic and Black has to look
for other possibilities instead of 4 (5).. d5}}}
\mainline{9... Nxd4 }
\xskakcomment{\small\texttt\justifying{\textcolor{gray}{Variant : \variation{9... Bd6} Might be more
interesting }}}
\mainline{10. hxg3 Qg4 }
\scalebox{0.90}{\chessboard}\\
\mainline{11. Qe1+ }
\xskakcomment{\small\texttt\justifying{\textcolor{darkgray}{[pgndiagram] This is a bad move
}}}

\xskakcomment{\small\texttt\justifying{\textcolor{gray}{Variant : \variation{11. Bf4} Would actually alow White to fight for the advantage }}}
\mainline{11... Be7 12. Bd3 }
\xskakcomment{\small\texttt\justifying{\textcolor{darkgray}{It is important to
defend the c2 pawn}}}

\xskakcomment{\small\texttt\justifying{\textcolor{gray}{Variant : \variation{12. Rh4} }}}
\mainline{12... Nf5 }
\xskakcomment{\small\texttt\justifying{\textcolor{darkgray}{And very quickly there comes
a serious mistake. Rh4 was threatened and the move defends against that, but
it disrupts the coordination of the Black's pieces}}}

\xskakcomment{\small\texttt\justifying{\textcolor{gray}{Variant n°1: \variation{12... Kf8} }}}

\xskakcomment{\small\texttt\justifying{\textcolor{gray}{Variant n°2: \variation{12... Nf6} }}}
\mainline{}
\scalebox{0.90}{\chessboard}\\
\mainline{13. Nf3 Bd7 }
\xskakcomment{\small\texttt\justifying{\textcolor{gray}{Variant : \variation{13... Kf8} Was slightly more resillient }}}
\mainline{14. Bf4 }
\xskakcomment{\small\texttt\justifying{\textcolor{darkgray}{[pgndiagram] Already it is hard to
find a move for Black. Ne5 is a terrible threat}}}
\mainline{14... f6 }
\xskakcomment{\small\texttt\justifying{\textcolor{gray}{Variant : \variation{14... Nf6} }}}
\mainline{}
\scalebox{0.90}{\chessboard}\\
\mainline{15. Ne4 }
\xskakcomment{\small\texttt\justifying{\textcolor{darkgray}{Now Nf2 is on the table as well}}}
\mainline{15... Ngh6 }
\xskakcomment{\small\texttt\justifying{\textcolor{darkgray}{A blunder in a lost position}}}

\xskakcomment{\small\texttt\justifying{\textcolor{gray}{Variant n°1: \variation{15... O-O-O} }}}

\xskakcomment{\small\texttt\justifying{\textcolor{gray}{Variant n°2: \variation{15... Kf8} }}}
\mainline{16. Bxh6 Nxh6 }
\scalebox{0.90}{\chessboard}\\
\mainline{17. Rxh6 }
\xskakcomment{\small\texttt\justifying{\textcolor{darkgray}{Winning the
piece or the queen}}}
\mainline{17... gxh6 18. Nxf6+ Kf8 }
\scalebox{0.90}{\chessboard}\\
\mainline{19. Nxg4 }
\xskakcomment{\small\texttt\justifying{\textcolor{darkgray}{[pgndiagram] An impressive
victory which allowed Steinitz to capture the crown.}}}
\mainline{}
\end{multicols}
\newpage

\end{document}